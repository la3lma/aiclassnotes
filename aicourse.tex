\documentclass[a4, 12pt, english, USenglish]{scrreprt}
% \usepackage{venn}
\usepackage[latin1]{inputenc}
\usepackage{makeidx}
\usepackage{pdftricks}
\usepackage[final]{pdfpages}

\usepackage{geometry, upgreek, booktabs, babel}
\usepackage[journal=rsc,xspace=true]{chemstyle}
\usepackage[version=3]{mhchem}
% \usepackage[footnotes]{notes2bib}
\usepackage[final]{microtype}
\usepackage[final, inactive]{pst-pdf}
\usepackage[colorlinks]{hyperref}

% equals with a "set" on top
\newcommand{\defeq}{\ensuremath{\stackrel{\mbox{set}{=}}}}

% XXX Should put a little arrow above its parameter.
\newcommand{\vectorXX}[1]{\ensuremath{#1}}

\newcommand{\fApartial}[1]{\ensuremath{\frac{\partial f}{\partial A_{#1}}}}
\newcommand{\jpartial}[1]{\ensuremath{\frac{\partial J}{\partial \theta_{#1}}}}
\newcommand{\thetaipartial}{\ensuremath{\frac{\partial}{\partial{\theta_i}}}}
\newcommand{\thetapartial}{\ensuremath{\frac{\partial}{\partial{\theta}}}}
\newcommand{\half}{\ensuremath{\frac{1}{2}}}
\newcommand{\sumim}{\ensuremath{\sum_{i=1}^{m}}}
\newcommand{\intinf}{\ensuremath{\int_{-\infty}^{\infty}}}
\newcommand{\Ft}{\ensuremath{{\cal{F}}}}
\newcommand{\ft}[1]{\ensuremath{{\cal{F}}({#1})}}

\newcommand{\sinc}[1]{\ensuremath{\mbox{sinc}{#1}}}
\newcommand{\screenshot}[2]{
\begin{figure}[htb]
\includegraphics[width=150mm]{screenshots/#1.jpg}
\label{#1}
\caption{#2}
\end{figure}}


\newcommand{\bb}[1]{\ensuremath{{\bf{#1}}}} % XXX Should be blackboard bold
\newcommand{\proj}[2]{\ensuremath{{\bb {#1}}_{#2}}}
\newcommand{\braces}[1]{\ensuremath{\left\{{#1}\right\}}}
\newcommand{\brackets}[1]{\ensuremath{\left\[{#1}\right\]}}
\newcommand{\parens}[1]{\ensuremath{\left({#1}\right)}}
\newcommand{\absval}[1]{\ensuremath{\left|{#1}\right|}}
\newcommand{\sqbraces}[1]{\ensuremath{\left[{#1}\right]}}
\newcommand{\commutator}[2]{\sqbraces{{#1}, {#2}}}


\newcommand{\dyad}[1]{\ensuremath{\ket{{#1}}\bra{{#1}}}}
\newcommand{\trace}[1]{\ensuremath{\mbox{tr}\, {#1} }}
\newcommand{\erfc}[1]{\mbox{erfc}\left(#1\right)}
\newcommand{\mXXX}[1]{\marginpar{\tiny{\bf Rmz:} {\it #1}}}
\newcommand{\celcius}{\ensuremath{^\circ}C}

\newcommand{\ev}[1]{\ensuremath{\left\langle{}#1{}\right\rangle}}
\newcommand{\ket}[1]{\ensuremath{\mid{}#1{}\rangle}}
\newcommand{\bra}[1]{\ensuremath{\langle{}#1{}\mid}}
\newcommand{\braKet}[2]{\ensuremath{\left\langle{}#1{}\mid{#2}\right\rangle}}
\newcommand{\BraKet}[3]{\ensuremath{\left\langle{}#1{}\mid{#2}\mid{#3}\right\rangle}}
\newcommand{\evolvesto}[2]{\ensuremath{{#1}\mapsto{#2}}}
\newcommand{\inrange}[3]{\ensuremath{{#1} \in \braces{{#2}, \ldots,{#3}}}}

\newenvironment{wikipedia}[1]
{
 {\bf From wikipedia: {\it #1}}
 \begin{quote}
}
{
 \end{quote}
}

\newcommand{\idx}[1]{{\em #1}\index{#1}}
\newcommand{\idX}[1]{{#1}\index{#1}}

\usepackage{url}
\newcommand{\tm}{\ensuremath{^{\mbox{tm}}}}
\newcommand{\aangstrom}{\AA{}ngstr\"{o}m{}\ }
%\newcommand{\aaunit}{\mbox{\AA}} % Just use A with ring, once encoding works properly
\newcommand{\aaunit}{\angstrom} % Just use A with ring, once encoding works properly
\newcommand{\munchen}{M\"unchen}
\newcommand{\zurich}{Z\"urich}
\newcommand{\schrodinger}{Schr\"odinger}
\newcommand{\ReneJustHauy}{Ren\'e-Just Ha\"uy}

%% Lavousier (with a lot fo weird spelling)


%% Crystallographic notation
%Coordinate
\newcommand{\crCoord}[3]{\mbox{\(#1,#2,#3\)}}
%Direction
\newcommand{\crDir}[3]{\mbox{\(\left[#1 #2 #3\right]\)}}
%Family of directions
\newcommand{\crDirfam}[3]{\mbox{\(\left<{}#1 #2 #3\right>\)}}
%Plane
\newcommand{\crPlane}[3]{\mbox{\(\left(#1 #2 #3\right)\)}}
%Family of planes
\newcommand{\crPlanefam}[3]{\left\{#1 #2 #3\right\}}

\newcommand{\oneCol}[2]{
  \ensuremath{\left(\begin{array}{r}{#1}\\{#2}\end{array}\right)}
}

\newcommand{\twoCol}[4]{
  \ensuremath{\left(\begin{array}{rr}{#1}&{#2}\\{#3}&{#4}\end{array}\right)}
}

%Negative number
\newcommand{\crNeg}[1]{\bar{#1}}

\makeindex

\begin{document}

\title{Lecture notes from the \\
online Artificial Intelligence course\\
tought by Peter Norvig and Sebatian Thrun \\
fall 2011}

\author{Bj\o{}rn Remseth \\ rmz@rmz.no}

\maketitle
\tableofcontents

% Comment out this in final version!

\parskip=\bigskipamount
\parindent=0pt.

\begin{abstract}

\end{abstract}

\chapter*{Introduction}

These are my notes from the course ``Artificial INtelligence'' taught
by Peter Norvig and Sebastian Thrun, fall 2011.

Unfortunately I lost my notes for module one and two in a freakish
accident involving emacs, to goats and a midget, so these notes start
from module three.

I usually watched the videos while typing notes in \LaTeX.  I have
experimented with various note-taking techniques including free text,
mindmaps and handwritten notes, but I've ended up using \LaTeX, since
it's not too hard, it gives great readability for the math that
inevitably pops up in the things I like to take notes about, and it's
easy to include various types of graphics.  The graphics in this video
is exclusively screenshots copied directly out of the videos, and to a
large extent, but not  completely, the text is based on Ng's
narrative.   I haven't been very creative, that wasn't my purpose.  I
did take more screenshots than are actually available in this text.
Some of them are indicated in figures stating that a screenshot is
missing.  I may or may not get back to putting these missing
screenshots back in, but for  now the are just not there.  Deal with
it .-)

A word of warning: These are just my notes.  They should't be
interpreted as anything else.  I take notes as an aid for myself.
When I take notes I find myself spending more time with the subject at
hand, and that alone lets me remember it better.  I can also refer to
the notes, and since I've written them myself, I usually find them
quite useful.   I state this clearly since the use of \LaTeX\ will
give some typographical cues that may lead the unwary reader to
believe that this is a textbook or something more ambitious.  It's
not.  This is a learning tool for me.  If anyone else reads this and
find it useful, that's nice. I'm happy,  for you, but I didn't have
that, or you in mind when writing this.   That said, if you have any
suggestions to make the text or presentation better, please let me
know.  My email address is la3lma@gmail.com.

Source code for this document can be found at github
\url{https://github.com/la3lma/aiclassnotes}, a fully processed pdf
file can be found at \url{http://dl.dropbox.com/u/187726/ai-course-notes.pdf}.


\chapter{Module three: Probability in AI}

\mXXX{It is a really good idea to go over those initial modules again
  before the final exam.  There will be questions, and it needs to be grokked.}

\subsection{Introduction}

Involved material.  Bayes networks.

Why won't your car start?. Perhaps flat battey.  Perhaps battery dead,
or it's not chargint, which in turn can come from the fanbelt or the
alternae being broken.  This set of relationship is a bayes network.

We can introduce things to inspect which can be used (lights, oi, gas,
dipstick) etc.  Causes affect measurements.


\screenshot{bayesnetwork1}{A Bayes network from the lectures}


Things we cannot measure immediately and measurements we can observe.
The bayes network helps us to reason with hidden causes based on
observable events.

A bayes network consists of nodes and arcs.  The non-observable nodes
are called random (or stochastic) variables.   The child of a node is
infered in a probabilistic way.

The graph structure and associated probability structure the network
is a compact representation of a very large joint probability
distribution.

With the network in place, we can compute probabilities of the random
variables.  We'll talk about how to construct bayes network and to use
them to reason about non-observable variables


Assumptions:

\begin{itemize}
\item Binary events
\item  Probability
\item  Simple bayes networks
\item  Conditional independence
\item  Bayes neworks
\item  D-separation
\item  Paramemeter coount
\item  later: Inferences based on bayes networks.
\end{itemize}

B-networks is really imporant, used in diagnostics, predictions,
machine learning.  Finance, google, robotics.  Also used as basis for
\idx{particle filters}, \idx{hidden markov models}, \idx{MDP}s and \idx{POMDP}s, \idx{Kalman
filters} and others.

\subsection{Probabilities}

Used to express uncertainty.   Given  a coin with probability for
coming u heads is  \( P(H) = 1/2\) the probaility of getting tails is 1/2.
The sum of the probabilities is always one.

Complementary probability (1-p).   

\subsection{Independence:}

\[X\bot  Y : P(X)P(Y) = P(X, Y) \]

where \(P(X)\) and \(P(Y)\) are called \idx{marginals}.

\subsection{Dependence:}

\subsubsection{Dependence:}

\subsubsection{Notation}

\[ P(X_1) = 1/2 \]

For one coin we have \(H: P(X_2=H | X_1=H) = 0.9\)  for another coin
we have \(H: P(X_2=T | X_1=T) = 0.8\)

The probability of the second coin being heads is 0.55.  The
calculation is  \(0.5 * 0.2 + 0.5 * 0.9 = 0.55\), and amazingly that
took me several minutes to figure out, and I needed to draw
a tree and do the sums to get the answer right.   It's obvious I need
this refresher course :-)


\subsubsection{Lessons}

The probability of some variable

\[
    P(Y) =  \sum_i P(Y|X=i) P(X=i)
\]

this is called the \idx{total probability}.    The negation of a probabilityu

\[
    P(\neg X | Y) = 1 - P(X|Y)
\]

\subsubsection{Cancer}

One percent cancer in the population.   Probability of a positive test
if you have the cancer \(P(+|C) =0.9 \).  The probability of  a
negative result if you don't have the cancer,  \(P(+|C) =0.1 \).

The probability of the test coming out positive if you don't have the
cancer is \(P(+|\neg C) = 0.2 \), the probability of the test being
negative if you don't have the cancer is \(P(-|\neg C) = 0.8 \).

Join probabilities (probability of having a positive test with cancer
and negative without (etc.)).

\subsubsection{The Bayes Rule. }

Thomas Bayes, presbytarian minister.  Bayes Theorem is:

\[
     P(A|B) = \frac{P(A|B)  \cdot P(A)}{P(B)}
\]

The terms are:  The \idx{likelyhood} = \(P(B|A)\),  the \idx{prior likelyhood} =
\(P(A)\), and the \idx{marginal likelyhood} = \(P(B)\).  \idx{the
  posterior} = \(P(A|B)\).

The interesting thing is the way that the probabilities are reverted.
We don't really care about the evidedence, we care if we have cancer
or not. 

The \idx{marginal likelyhood} is in turn denoted as the \idx{total
  probability} of \(B\): 

\[
    P(B) = \sum_a P(B|A=a) P(A=a)
\]

In the cancer case we are interested in:

\[
\begin{array}{lcl}
P(C|+) &=&\frac{P(+|C)\cdot P(C)}{P(+)}  \\
           &=& \frac{0.9 \cdot 0.01}{(0.9 \cdot 0.01) + (0.2 \cdot 0.99)}   \\
           &=& \frac{0.009}{0.009+ 0.198} \\
           &=& 0.0434
\end{array}
\]

The Bayes Rule can be drawn graphically. Two variables.   B
observable, A not.  A causes B to be positive with some probability
\(P(B|A)\).  What we care about is \idx{diagnostic reasoning} which is
the inverse of the \idx{causal reasoning}.   The graphical
representation is a graph with an arrow going from A to B.

To fully specify a Bayes network with one observable and one
non-observable, we need three parameters: The observed probability (of
the observable), the conditional probability of the non-observable
being true in the case of the observable being true, and the
probability of the non-observable being true when the observable is
false.





\subsubsection{Computing bayes rule}

Given the rule 

\[
     P(A|B) = \frac{P(A|B)  \cdot P(A)}{P(B)}
\]

the denominator is relatively easy to calculate since it's just a
product.  The total probability (the divisor) is really hard to
compute since it is a sum over potentially many terms.  However, the
denominator isn't dependent on the thing we're trying to find the
probability for.  If for instance we are interested in 


\[
     P(\neg A|B) = \frac{P(\neg A|B)  \cdot P(\neg A)}{P(B)}
\]

we use the same denominator.  We also know that \(P(A|B) + A(\neg A|B)
= 1\) since these are \idx{compementary events}.  This allows us to
compute the bayes rule very differently by ignoring the normalizer.

First we calculate \idx{pseudoprobabilities}

\[
\begin{array}{lcl}
 P'(A|B) &=& P(B|A) P(A) \\
 P'(\neg A|B) &=& P(B|\neg A) P(\neg A) \\
\end{array}
\]

then we normalize by multiplying the pseudoprobabilities with a
\idx{normalizer} (eta)

\(\eta = \parens{P'(A|B) + P'(\neg A | B)}^{-1} \)

We defer calculating the actual probabilities:

\[
\begin{array}{lcl}
 P(A|B)&=&  \eta P'(A|B)  \\
 P(\neg A|B) &=& \eta P'(\neg A|B)\\
\end{array}
\]


\subsubsection{Two cancer}

\screenshot{twocancer}{A Bayes network from the lectures}



\subsection{Conditional Independence}

In the two-cancer example (to be scanned and/or type) we assumed
conditional independence:

\[
P(T_2| C_1T_1) = P(T_2| C) 
\]


Conditional independence is a really big thing in bayes networks.
The thing is that if an arrow goes out of something, that will make
variables in the receiving end dependent.  It's even better, since
it's a model of the causality behind the dependence.
Snippets


Absolute indepence does not give conditional independence, conditional
independce does not imply absolute independence.

Differen type of bayes network.  Two hidden causes that are confounded
into a single state.  Happyness may be caused by sunnyness, a raise.

The ``explaining away'' method.  If there are multipe causes for
happiness.  If we can observe one of them, that makes the other less
likely, and it is ``explained away''.

\screenshot{compactbayes}{Compact bayes}


Bayes networks defines probabiliteis over graphs of random variables.
Instead of enumerating all dependenies, the network is defined by
distributions of probabilities dependent on the incoming arcs.  The
joint distribution of probability of the network is defined as the
product of all the product of all the nodes (dependent of the nodes).
This is a very compact definition.  It doesn't have to look at the set
of all subsets of dependent variables.   In this case only ten
parameters instead of 31.  This compactness is the reason bayes
networks are used in so many situations.


Unstructured joint representations sucks compared with bayes
representation.

\subsection{Parameter count}

A bayes network is defined as an equation like this one.

\[
   P(A,B,C) = P(C| B, C) * P (A) * P(B) = \prod_i P_i(c_i  | c_{1} \cdots c_{n(i)})
\]

The number  of parameters that are necessary to specify this network
is \(\sum_i 2^{n(i)} \), so in the case above the count would be \(1 + 1
+ 4 =  6\). 



\subsection{D-separation}

\screenshot{dseparation}{D separation}





Two variables are independent if they are independent of  Any two
variables independent if they are linked by just unknown variables.
Anything downstreaam 

\screenshot{dseparation-1}{More D separation}



Reachability = D-separation

\idx{Active triplets}:  render triplets dependent.


\idx{Inactive triplets}: render triplets independent.

\screenshot{reachability}{Reachability}


We'vw learned about graph structure, compact representation,
conditional indepence and some applications.

\section{Probabilistic inference}


Probability theory and bayes net and independence.  In this  unit we
will look at probabilistic inference.

In probabilisticinference we don't talk about \idx{input variables} and
\idx{output values}, but evidence and query.  The \idx{evidence} is
the stuff we know something about, the \idx{query} are the things we
want to find out about.  Anything that is neither evicence nor query
is a \idx{hidden variable}.  In probabilstic inference, the output
isn't a single number, but a \idx{probability distribution}.  The
answer will be a \idx{complete probability distribution} over the query
variables. We call this the \idx{posterior distribution}.

Some notation:

The \(Q\)s are the query variables, the \(E\)s are the evidence
variables.  The computation we want to come up with is:

\[
  P(Q_1, Q_2, \ldots | E_1= e_1, E_2=2_2, \ldots)
\]

Another question we can ask: What is the most probable distribution.
The notation for that is:

\[
  \mbox{argmax}_q\, P(Q_1=1_1, Q_2=1_2, \ldots | E_1= e_1, E_2=e_2, \ldots)
\]


Find the combination of q values that are ``maxible'' given the
evidence.  That is the set of qs having the highest probability of
being true.

In an ordinary computer language computation goes only one way, from
input to output.  Bayes nets are not restricted to going in only oen
direction.  We can go in the causal directing, using the evidence as
input and using the query values at the bottom to figure out what's
the most likely situation given some evidence.  

However we can reverse the \idx{causal flow}.  We can actually mix the
variables in any way we want.

inference on bayes networks: Enumeration

Goes through all the combinations, adds them up and comes up with an
answer:

\begin{itemize}

  \item {\em State the problem:} ``What is the probability that the burglar alarm
went of given that john called and Mary called'': \( P(+b | +j, +m)
\).   

\item We will be using a definition of conditional probability that
goes like this:

\[
P(Q|E) = \frac{P(Q,E)}{P(E)}
\]

\item  The query we wish to get is the \idx{joint probability distribution}
  divided by the \idx{conditionalized variables}.

\[
 P(+b|, +j, +m) = \frac{P(+b, +j, +m)}{P(+j, +m)}
\]

Reading this equality from left to right, it can be interpreted as
rewriting the conditioned probability into a fraction of unconditioned
probability.

\end{itemize}

We're using a notation here:

\[
P(E=\mbox{true}) \equiv  P(+e) \equiv 1 - P(\neg e) \equiv P(e)
\]

The latter notation may be confusing, since it's unclear if \(e\) is a variable.

First take a conditional probability, and then rewrite it as an
unconditional probability.  Now we enumerate all the atomic
probabilities and calculate the sum of products.

We'll take a look at \(P(+b, +j, +m) \).  The probability of these
three terms can be expressed like this.

\[
  P(+b, +j, +m)   = \sum_e \sum_a P(+b, +j, +m)
\]

We get the answer by summing over all the possibilities of \(e\) and \(a\)
being true and false, four terms in total.   To get the values of
these  atomic events, we need to rewrite the term \(P(+b, +j, +m)\) to
fit the  conditional probability tables that we have associated with
he bayes network.

\screenshot{enumeration}{Enumeration}


\[ 
\begin{array}{lcl}
  P(+b, +j, +m)   &=&  \sum_e \sum_a P(+b, +j, +m) \\
&=& \sum_e \sum_a P(+b)P(e) P(a|+b, e) P(+j|a) P(+m, a) \\
&=& f(+e,+a)  + f(+e,\neg a) + f(\neg e,+a) + f(\neg e, \neg a)
\end{array}
\]

The numbers to fill in we get from conditional probability tables


\screenshot{enumerationtables}{Enumeration tables}



For simple networks enumeration is simple to do.  For five hidden variables
there will only be 32 terms to consider.   If there are several tens of
hidden variables there will be billions or quadrillion rows.  That is
not just practical.  

\subsubsection{Pulling out terms}

The first optimization we'll consider is \idx{pulling out terms}.  If
we start with:

\[
  \sum_e \sum_a P(+b)P(e) P(a|+b, e) P(+j|a) P(+m, a)
\]

The probability of   \(P(+b)\) will be the same all the time, so we
can pull it outside the summation.  That's a little bit less work to
do.   We can also move the term \(P(e)\)  can be moved in front of the
second summation: 

\[
   P(+b)\sum_e P(e) \sum_a P(a|+b, e) P(+j|a) P(+m, a)
\]

This is good since it means that each row takes less work, but it's
still a large number of rows.

\subsubsection{Maximize independence}

For instance a network that is a linear string can have inference done
in time that is proportional to the number \(O(n\), whereas a network
that is  a \idx{complete network}, i.e. a network where all nodes are
connected, could take time proportional to \(O(2^n)\) for boolean
variables.   In the alarm network we took care that we had all the
dependence relations represented in the network.  However, we could do
it differently.

The moral is is that bayes networks are written the most compactly
when they are written in that \idx{causal direction}, when the networks
flows from causes to effects.
In the equation

\[
   P(+b)\sum_e P(e) \sum_a P(a|+b, e) P(+j|a) P(+m, a)
\]

we sum up everything.  That's slow since we end up repeating a lot of
work.  We'll now look at another technique called \idx{variable
  elimination}  which in many network operates much faster.  It's
still a hard problem ( \idx{NP-hard}) to inference on Bayes networks
in general, but variable elimination works faster in most practical
cases.  It requires an algebra for addressing elements in \idx{multi
dimensional arrays} that comes out of the \(P\) terms in the
expression above.

We look at a network with the variables R (raining), T (Traffic).  T
is dependent on R.   L (late for next appointment). L is dependent on
T.  For a simple network like this enumeration will work fine, but we
can also use eleminitation.

It gives us a way to combine parts of the network into smaller parts,
en then enumerate over those smaller parts and then continue
combining.  So we start with a big network we elminiate som of the
variables, then compute by \idx{marginalizing out} and then we have a
smaller network to deal with.

\screenshot{variable-elimination}{Variable elimination}

\begin{enumerate}

\item{\em Joining factors:}  We chose two or more of the factors, and
  join the together to a new factor.  \(P(R,T)\) could be one such
  factor.  It's just a ``join''operator over the relevant tables where
  the probabilities are multiplied. I'm sure an sql statement can be
  concocted to calculate this :-)

\item{\em summing out or marginalizing:}   Continue the process to 
  remove factors, so in the end you end up with a much smaller number
  of factors.

\end{enumerate}

If we make a god choice of the orde of eliminations to do it can be
very much more efficient than just enumeration.

\subsection{Approximate inference sampling}

Repeat experiments to sample input.   Repeat until we can estimate the
joint probability.  Large numbers of samples are necessary to get true
to the true distribution.

Sampling has an advantage over inference in that it gives us a
procedure to get to an approximation to the joint distribution,
whereas the exact computation may be very complex.

Sampling has another advantage: If we don't know what the probability
distributions are, we can still proceed with sampling, but we couldn't
do that with inference.


\screenshot{samplingexample}{Sampling example}


Four variables, all boolean. Cloudy, Sprinkler, Rain, WetGrass.   To
sample we use the probability tables and traverses the graph and
assigns values randomly (according to the distributions) and then uses
those values to build up an image of the total probability.

The probability of sampling a particular value depends on the parent.
In the limit the count of the counted probability will approach the
true probability.  With an infinite number of samplings we would know
the result :-)  The sampling method is \idx{consistent}.  We can use
this sampling method to compute the complete joint probability
distribution, or we can use it to compute values for an individual
variable.


Now what if we wish to compute a conditioal probability?    To do that
we need to to start doing the same thing (generating samples), but
then reject the samples that don't match the condition we are looking
for.  This technique is called \idx{rejection sampling} since we
reject the samples that don't have the properties we are interested in
and keep those that do.

This procedure would also be consistent.

Tere is a problem with rejection sampling: If the evidence is
unlikely, you end up  rejecting a lot of samples.  Let's consider the
alarm network again.  If we're interested in calculating the
probability \(B|a+)\) i.e. the probability of a burglary given that
the alarm has gone off.  The problem is that burglaries are very
infrequent, so we will end up rejecting a lot of samples.

To counter this we introduce a new method called \idx{likelyhood
  weighting} that generates samples that can be used all the time.  We
fix the conditional variables (setting \(a+\) in this case), and then
sample the rest of the variables.

We get samples that we want, but the result we get is
\idx{inconsistent}.  We can fix that by assigning a probability to
each sample and weighting them correctly.

In \idx{likelyhood weighting} we sample like before but add a
probabilistic weight to each sample.  Each time we are forced to make
a choice, we must multiply a probability for the value we choose.
With weighting likelyhood weighting is consistent.

Likelyhood weighting is a great technique but it doesn't solve all
problems.   There are cases where we fix some variables, but not all,
and that makes variables that the fixed variables are dependent on
likely to make a lot of values with very low probabilities. It's
consistent, but not efficient.

\idx{Gibbs sampling} takes all the evidence, not just the upstream
evidence into account.  It uses a technique called \idx{markov chain
  monte carlo} or \idx{mcmc}.  We resample just one variable at a
time, conditioned on all the others:  Assume we have a set of
variables and initialize them to random values keeping the evidence
values fixed.   At each iteration through the loop we select just one
non-evidence variable and resample that based on all the other
variables.  That will give us another variable.   Repeat.


\screenshot{gibsssampling}{Gibbs sampling}

We end up walking around in the space.  In mcmc all the samples are
dependent on each other, in fact adjacent samples are very similar.
However, the technique is still consistent.

\idx{The monty hall problem}  Three doors, one with an expensive
sports car, the other two contains a goat.   If you choose door number
one, the host will now open a door containing a goat.   You can now
stick with your choice or switch.







\chapter{Machine learning}

The world is data rich.  Chemihal, financial pharmaceutical databases
etc.  To make sense of it macine learning is really important to make
sense of it.

So ar we've talked about bayes networks that are known.  In machine
learning addreses how to learn models from data.  

\section{Supervised learning}

\screenshot{stanley}{Stanley, the winning robot in the DARPA
  grand challenge.}

Thrun shows off the Stanley car  that uses a laser that builds a 3d
model of the terrain ahead.    It tries to drive on flat ground.  The
laser only looks about 25 meters ahead.   The camera image sees
further, but the robot uses machine learning to extrapolate the
knowledge about where drivable road is loacated out to two hundred
meters, and that was crucial in rder to get the robot to drive fast.

\begin{itemize}

\item what?  Parameters, structure and hidden concepts (some hidden
 rules may exist in the data).

\item What from?   Some sort of target information. In supervised
  learning we have specific target labels
  rules) .
\item  Unsupervised learning? Target labels are missing.
\item Reinforcement learning.   Receiving feedback from the
  environment
\item what for? Prediction, diagnosis? Summarization (gisting) ..
\item how? Passive (if the agent is just an observer), otherwise
  active   Sometimes online (when data is generated) or offline.
\item Classification (fixed no of classes) v.s. regression (continous)
\item Details?  Generative v.s. discriminative.  Generative tries to
          describe stuffs as general as ppossible, dicriminitave 
          attempts to find things as specific as possible.
\end{itemize}

In supervised learning we have a bunch of features and a target:
\[
   x_1, x_2, \ldots , x_n \rightarrow y
\]


\screenshot{supervisedlearning}{Supervised learning}

The learning algorithm gets a learning set and generates a model.
Features, behavior etc.  The subject is to identify the function
\(f\).

\screenshot{errorclasses}{Error classes}

When selecting models, \idx{occam's razor} is good: Everything else
being equal choose the less complex hypothesis.  In reality there is a
tradeoff between fit and low complexity.   \idx{bias variance
  methods}.

In practice it's a good idea to push back on complexity to avoid
overfitting and generative errors.   \idx{Overfitting} is  a major source of
poor performance for machine learning.

\subsection{Spam detection using bayes networks}

\screenshot{spamproblem}{Detecting spam in email}

We wish to categorize into spam/ham.  to get things usable for emails
we need a representation.  We use the \idx{bag of words}
represention.   It is a word frequencey vector.  The vector is
obvlivious to the location of the word in the input.

In the ham/spam examle,  this the vocabulary 

\begin{figure}[htb]
\begin{verbatim}
click 
costs
event
is
link
money
offer
play 
secret
sport
sports 
today
went
\end{verbatim}
\label{vocabulary}
\caption{The spam/ham vocabulary}
\end{figure}


\screenshot{spamham}{Calculating the probability of spam}

the size is 14.   The probability that is  3/8 that it's spam

\screenshot{maxlikelyhood}{First maximum likelyhood slide}
\screenshot{maxlikely2}{Second maximum likelyhood slide}


\subsection{Maximum likelyhood}

Since I'm not really conversant on maximum likelyhood issues, I'll
copy a bit in detail, since that is something that frequently helps me
transfer the subject matter into my mind :-)

We assume that there are two possible outcomes of a process, ``S'' and
``H'', and then define \(P(S) = \pi\) and conversely \(P(H) = 1 - \pi
\). Rewriting this yet once more gives:

\[
p(y_i) =
\left\{
\begin{array}{cclcl}
   \pi & \mbox{if} & y_i &=& S \\
1-\pi & \mbox{if} & y_i &=& H \\
\end{array}
\right.
\]


Now comes a bit that I don't understand, since he says:

\[
p(y_i) = \pi^{yi} \cdot (1-\pi)^{1 - y_i}
\]

Less mysterious, since it is a a consequence of the above:

\[
p(\mbox{data}) = \pi_{i=1}^n p(y_i) = \pi^{\mbox{count}(y_i=1)} \cdot
(1 - \pi)^{\mbox{count}(y_i=1)} 
\]



\[
\begin{array}{lclcl}
  P(\mbox{secret}| \mbox{spam})  &=& 3/9 &=& 1/3 \\
  P(\mbox{secret}|\mbox{ham})    &=& 1/15 && \\
\end{array}
\]


\subsection{Relationship t Bayes Networks}

We are making a maximum likelyhood etimator modelled as an bayesian
estimator and using machine learning to finding the network.

Given a twelve word vocablary, we need 23 parameters. P(spam), and two
for each word (one for ham, and one for spam). Now, there are 12
words, so that makes 12 parameters for ham probability, but since
these add up to one, we only need eleven :-)  The same thing applies
to spam, so that is 11 + 11 + 1 = 23.

So, what i the probability of classifying ``sports'' as spam?  Well,
we have to bring out Bayes:

\begin{verbatim}
\[
P(spam|''sports'') =
    \frac{p("sports"|spam) \cdot p(spam)}{P(spam)} 
= 
    \frac{p("sports"|spam) \cdot p(spam)}{P(m|spam)P(spam) + P(m|ham)P(ham)}
\]
\end{verbatim}

1/9*3/8  / (1/3 * 3/8   + 1/3 * 5/8) = (3/72)/ (18/72) = 3/18 = 1/6 = 0.667
The answer is 3/18

\screenshot{overfittingbayes}{Overfitting using a naive Bayes classifier}

\subsubsection{Laplace smoothing}

One way of fixing the overfitting is \idx{laplace smoothing}. 

\[
\begin{array}{cccc}
\mbox{ML}&p(x) &=& \frac{\mbox{count}(x)}{N} \\
\mbox{LS}(k)&p(x) &=& \frac{\mbox{count}(x) + k }{N + k|x|} 
\end{array}
\]

ML = maximum likelyhood.  LS = Lapace smoothing

where \(\mbox{count}(x)\) is the number of occurrences of this value of the
variable \(x\).  \(|x|\) is the number of values that the variable \(x\) can take
on.  \(k\) is a smoothing parameter.  And \(N\) is the total number of
occurrences of \(x\) (the variable, not the value) in the sample
size.   

This is equivalent of assuming that we have of fake count \(k\) to he
count, and then adding \(k\) to every single class we are estimating over.


if k=1. one message with one spam.  For the laplace smoothed thing we
get these nmbers

\screenshot{laplacesmoothingexample}{An example of using Laplace smoothing}

\screenshot{hamspamlapace2}{Using Laplace smooting in a spam detector}
\screenshot{hamspam3}{Calculating even more spam probabilities using
  Laplace smoothing}


\subsubsection{Summary naive bayes}

We've got features and labels (e.g. spam/ham).  We used ML and
laplacian smoother to make  a bayes network.      This is called a
\idx{generative model} in that the conditioned probabilities are all 
aiming to maximize the predictability of individual features as if
those features describe the physical world.  We use the \idx{bag of
  words} model.   

This is a very powerful method for finding spam.  It's unfortunately
not good enough since spammers know how to circumvent naive bayes
models.

\subsubsection{An advanced spam filter:}

\begin{itemize}
\item    Known spamming ip
\item   Have you mailed the person before?
\item   Have 1000 other people received the same message?
\item   Is the email header consistent?
\item   Is the email all caps?
\item   Do inline URLs point to the points they say they point to 
\item   Are you addressed by name?
\end{itemize}

All of these features can be used by a  naive bayes filter.

\subsubsection{Hand writen digit recognition}

Handwritten zipcodes. The machine learning problem is to take a
handwritten symbol and find the matching correct number.   The input
vector for the naive bayes classifier could be pixel values in a 16x16
grid.   Given sufficient many training examples we could hope to
recognize letters.  However, it is not sufficient \idx{shift
  invariant}.  There are many different solution , but one could be
smoothing.  We can convolve the input to ``smear'' the input into
neighbouring pixels with a gaussian variable, and we might get a
better result.

However, Bayesian classification is not really a good choice for this
task.  The indipendent variable assumption for each pixel is too
strong an assumption in this case, but it's fun :-)


\subsubsection{Overfitting prevention}

Occam's razor tells us that there is a tradeoff between precision and
smoothnes.  The ``k'' parameter i Laplacian smoothing is that sort of
thing.  However, it gives us a new question: How to choose the ``k''.   

One method that can help is \idx{cross validation}.  The idea is to
take your training data and divide it into three buckets: Train, Cross
validate and Test.  A typical percent-wise partitioning will be 80/10/10.

\screenshot{crossvalidation}{Cross validation}

First train to find all the parameters.   Then for the cross
validation data test how well the classifier works.  In essence
perform an optimization of the classifier where the cost function is
the match on the cross validation data, and use this to optimize the
smoothing parameters.  Finally, and only once do we validate the
result using the test data, and it's only the performance on the test
data we report.

It's really important to use different test and crossvalidation sets
different, otherwise we are prone to overfitting.  This model is used
by pretty much used by everyone in machine learning.    Often one
mixes the training and crossvalidation sets in different ways.  One
common way is to use ten mixes, called ``tenfold cross validation''
(and run the model ten times).


\subsubsection{Supervised learning, regression}

We have classification and regression problems.  Regression problems
are fundamentally different than classifications.   Bayes networks
only predicts discrete classes so it's not useful for regression.

In regression our job is to fit a curve to predict the line.

In all regression we have a set of data points that maps to a
continous variable, and we look for a function f that matches the
vector x into y. 

\screenshot{linearregression}{Linear regression}

We are trying to minimize the loss function.  In this case the loss
function is the sum of all the errors (squared).

This system can be solved in closed form using the \idx{normal
  equations}.

To find the minimum we take the derivative of the quadratic loss
function

\screenshot{minimizingquadraticloss0}{Minimizing quadratic loss}

\screenshot{minimizingquadraticloss}{Hand calculating the parameters
  of linear regression}

\screenshot{linearregressionquiz}{An actual calculation of a set of
  linear regression parameters}

\screenshot{linearregressionformulahand}{The linear regression
  formulas in my beautiful handwriting}

When calculating these things, I fin that it's a good idea to
calculate all the sums in advance in a table, and then just plug them
in.  When using a computer it's of course even simpler, but we're not
doing that now.


Linear regression works very well when the data is linar, but not
otherwise :-) 

\screenshot{poorlinearregression}{Some curves that are not very well
  approximated by linear regression}

Outliers are handled very badly.  Also when x goes to infinity, your y
also does that.  That is bad.

Logistic regression is an alternative.

\[
     z = \frac{1}{1 + e^{f(x)}}
\]

\subsubsection{Regularization}

\screenshot{regularization}{Regularization imposes a cost for
  introducing parameters.  This is good since it reduces the risk of overfitting.}
We assume that loss is a combination of loss from data and loss from
parameters:

\[
\begin{array}{lclccc}
   \mbox{Loss} &=&    \mbox{Loss}(data) &+&   \mbox{Loss}(parameters)\\
   &&\sum_j\parens{y_j - w_ix_j-w_0}^2 &+& \sum_i\parens{w_i}^p
\end{array}
\]

The \idx{parameter loss} is just some function that penalizes the
parameters becoming large, a potential like shown above (and it's
usually one or two)

 
\subsection{Minimizing more complicated loss functions}

\screenshot{gradientdescent}{Gradient descent}

In general there are no closed form solutions, so we have to revert to
iterative methods, often using gradient descent.


Many books on optimization can tell you how to avoid local minima, but
that isn't part of this course.

\screenshot{graddescentalgorithm}{The Gradient descent algorithm}

\subsection{The perceptron algorithm}

A simple algorithm invented in the forties.  The perceptron finds a
linear separator.  We can create a linear separation classification
algorithm:

\screenshot{perceptronalorithm}{The perceptron algorithm}

The altorithm only converges if the data is separabel, but then it
converges to a linear separator

\begin{itemize}

\item Start with a random guess for \(w_1, w_0\).  It's usually
    inaccurate.

\item Update the new weight \(m_i^m \leftarrow w_i^{m-1} +\alpha\parens{y_j - f(x_j)}\)
The learning is guided by the difference between the wanted result.
    inaccurate. It's an online rule, it can be iterated many times.

\end{itemize}

\screenshot{perceptronupdate}{Perceptron updates}

Use the error to move in the direction that minimizes the error.    A
question is determine which of the many possible linear separators
should be preferred.

\subsubsection{K-nearest neigbours}

The final method is a non-parametric method.  In parametric learning
the number of parameters is independent of the learning set.  In
parametric methods we kind of drag the entire learning set (or a
subset of it) along into the model we are creating.  For the Bayes
examples since the dictionary was variable, so that that wasn't exactly right, but for any fixed dictionary the
number of the parameter number is independent the training set size.

1-nearest neigbour is a very simple method.  Given a set of data
points search for the nearest point in the euclidian space and copy
its label.  The \idx{k-nearest neigbour} is also blatantly simple.  In
the learning step, just memorize all the data.   When an input data
arrives do, you find the k nearest neigbours, and you return the
majority class label.  It's actually brutally simple.  The trick is of
course to pick the right ``k''.    K is a smoothing parameter (like
the lambda in the laplace smoothing).   Higher k gives more
smoothing.   The cleaner the decision boundary will be, but the more
outliers there will be.  k is a \idx{regularizer}.  Like for the
laplace smoother we can use cross validation to find an optimal value
of it.

KNN has two main problems, the first is very large data sets, and the other
is very large feature spaces.  Fortunately there are methods of
searching very quickly in tres so the search ``\idx{kdot trees}''.
Large feature spaces is actually the bigger problem.  The tree methods
become brittle for higher dimensions.  

\screenshot{knngraphlength}{The graph length as a function of the
  dimensions in the dat set. Conclusion: K-means don't work that well
  for higher order data.}

The edge length goes up really fast. All points ends up being very far
away.  For few dimensions like three or four it works well, for twenty
or hundred it doesn't.











\section{Unsupervised learning}

Here we just have data,  so our task is to find structure in the
data.

\idx{IID} \idx{Identically distributed and Independently drawn} from the same
distribution.  We wish to perform a \idx{density estimation} to find the
distribution that produced the data.

Methods used are \idx{clustering} and \idx{dimensionality reduction}.
Unsupervised learning can be applied to find structure in data.  One
fascinating application is \idx{blind signal separation}.  How to
recover two speaker output from a single microphone. It's possible,
but it doesn't require target signals.   It can be construed by using
\idx{factor analysis}.

Unsolved problem from Streetview.  The objects that can be seen is
similar in many of the images.   Can one take the streetview dataset
to discover concepts such as trees, stopsigns, pedestrians etc.
Humans can learn without target sets, can machines?

Clustering: The most basic forms of unsupervised learning \idx{k-mans}
and \idx{expectation maximization}.

\subsubsection{k-means}

\begin{itemize}
\item Choose to cluster centers in the space and do so at random.   
\item The clusters will now divide the sample space in two by a line
  being orthogonal to the vector between the two cluster centers (a
  \idx{voronoi graph} based on the two centers).
\item Now, find the optimal cluster centers for the two points.
Minimize the joint squared distance from the center to all the data
points in the cluster.

\item Now iterate.   The cluster centers have moved, so the voronoi
  diagram is different, that means that the clustering will be
  different.  Repeat until convergence.

\end{itemize}

\screenshot{kmeansalgorithm}{The K-means algorithm in a nutshell}


The algorithm is know to converge.  However the general clustering
algorithm is know to be \idx{NP-complete}.

Problems with k-means: First we need to know k.     Then there are the
local minima we can fall into. There is a general problem with high
dimentionality and there is a lack of mathematematical basis for it.



\subsection{Expectation maximization}

Probabilistic generalization of k-means..  Uses actual probability
distribution. It has a probabilistic basis, more general than k-means.

\subsubsection{Gaussians}

Continous distribution .  Mean is \(\mu\) the normal distribution
\(\sigma\) (where the second derivative changes sign :-)  The density
is given by:

\[
   f(x) = \frac{1}{\sqrt{2\pi}}\exp{\parens{-\frac{1}{2}\, \frac{(x-\mu)^2}{\sigma^2}}}
\]

This distribution has the nice property (like all continous
probability distributions) that:

\[
    \int_{-\infty}^{\infty} f(x) dx = 1
\]

Goes to zero exponentially fast :-) Very nice.

For every  we can now assign a density value.  For any interval we can
now generate probabilities by calculating integrals.


The multivariate gaussian has multipel input variables. Often they are
drawn by level sets.  The center has the highest probability.


The formula is;
\[
    (2\pi)^{-\mu/2}
    \left|\sum\right|^{-\frac{1}{2]}}\exp\parens{-\frac{1}{2}
       \parens{x-\mu}^T\sigma^{-1}(x - \mu)}
\]

\(\sigma\) is a \idx{covariance matrice}. \(x\) is a \idx{probe point}
and \(\mu\) is a \idx{mean vector}.  This can be found in any textbook
about \idx{multivariate distributions}.  It's very simillar: Quadratic
function, exponential, normalization etc.


\subsubsection{Gaussian Learning}

In the one dimensional case the \idx{Gaussian} is parameterized by the
\idx{mean} \(\\mu\) and the \idx{variation} \(\sigma^2\), so we have:

\begin{verbatim}
\[
    f(x | \mu, \sigma^2) =     \frac{1}{\sqrt{2\pi}\exp{\parens{-\frac{1}{2}\, \frac{(x-\mu)^2}{\sigma^2}}} 
\]
\end{verbatim}


Fitting data, if we assume that the data is one  dimension.  There are
really easy formaula for fitting gaussians to data:

\[
\begin{array}{lcl}
\mu &=& \frac{1}{M}\sum_{j=1}^M x_j \\
\sigma^2 &=& = \frac{1}{M}\sum_{j=1}^M (x_j - \mu)^2 \\
\end{array}
\]

The mean and the sum of the squared deviations from the mean.

For multiple data points we have for data \(x_1, \ldots, x_M\)

\[
\begin{array}{lcl}
p(x_1, \ldots, x_M|\mu, \sigma) &=& \prod_i f(x_i|\mu, \sigma) \\
&=&\parens{\frac{1}{2\pi\sigma^2}^{M/2}} \exp{\parens{-\frac{\sum_i(x_i - \mu)^2}{2\sigma^2}}}
\end{array}
\]

The best choice for a parametrization is the one that maximizes the
expression above.  That gives us the \idx{maximum likelyhood
  estimator}.  We now apply a trick, which is to maximize the log of
the function above:

\[
M/2 \log\frac{1}{2\pi\sigma^2} - \frac{1}{2\sigma^2}\sum_{i=1}^M(x_i - \mu)^2
\]

The maximum is found by setting the partials to zero:

\[
0 = \frac{\partial\log f}{\partial\mu} = \frac{1}{\sigma^2} \sum(x_i-\mu)
\]

This is equivalent to \(\mu = \frac{1}{M}\sum_{i=1}^{M}x_i\), so we
have shown that the mean is the maximum likelyhood estimator for
\(\mu\) in a gaussian distribution :-)

We do the same thing for the variance:

\[
\frac{\partial\log f}{\partial\sigma} = -\frac{M}{\sigma}
+\frac{1}{\sigma^3}\sum_{i=1}^{M}(X_i - \mu)^2 = 0
\]

which works out to \(\sigma^2 = \frac{1}{M}\sum_{i=1}^M(x_i-\mu)^2\)

\mXXX{Verify these steps sometime}

\subsection{Expectation maximization}

After this detour into Gaussians we are ready to  continue with to \idx{expectation maximization} as a generalization of \idx{k-means}.

Stat the same way. Two randomly chosen cluster centers.  But instead   
of making a \idx{hard   correspondence} we make a \idx{soft
  correspondence}. 
A data point atracted to a cluster according the the \idx{posterior
  likelyhood}.  The adjustment step now corresponds to all data
points, not just the nearest ones.  As a result the cluster center
tend not to move so far way.  The cluster centers converge to the same
locations, but the correspondences are all alive. There is not a
zero-one correspondence.

Each data point is generated from a \idx{mixture}.   There are K
clusters corresponding to a class. We draw one at random

\[
p(x) = \sum_{i=1}^M p(c=i) \cdot P(x|c=c=i)
\]

Each cluster center. has a multidimensional gaussian attached. The
unknowns are the probabilities for each cluster center \(\pi_i\) (the
\(p(c=i)\) s and the \(\mu_i\) and in the general case \(\mu_i\sum_i\)
for each individual Gaussian.


The algorithm:  We assume that we know the \(\pi_i\) and
\(\sigma_i\)s.   The E-step of the algorithm is:

\[
    e_{i,j} = \pi_i \cdot (2\pi)^{\mu/2}\left|\sum\right|^{-i} \exp\parens{\frac{1}{2}}\parens{x_j-\mu_i}^T{\sum}^{-1}\parens{x_j-\mu_i}
\]

\(e_{i,j}\) the probability tha the \(j\)-th data point belongs to
   cluster center \(i\). 

In the M-step we figure out what these probabilities should have been.

\[
\begin{array}{lcl}
  \pi_i &\leftarrow& \sum_j e_{ij}/M \\
 \mu_i &\leftarrow& \sum_j  e_{ij} x_j / \sum_je_{ij} \\
  \sum_i &\rightarrow& \sum_j  e_{ij}(x_j - \mu_i)^T(x_j-\mu_i) /
  \sum_j e_{ij}
\end{array}
\]

These are the same calculations that we used before to find aussians,
but they are weighted.

But how many clusters? In reality we don't know that and will have to
find out.    The justification test is based of a test based on the
negative log likelyhood and a penalty for each cluster.  In particular
you want to minimize this function

\begin{verbatim}
\[
  -\sum_j\log p\parens{x_j|\sigma_\sum_i k} + \mbox{cost}\cdot k
\]
\end{verbatim}

We maxzimize the posterior probability of data.  The logarithm is a
monotonic function and the negative term in front is to make the
maximization problem into a minimization problem.  

We have a constant cost per cluster.  The typical case is to guess an
initial k, run the EM; remove unecessary clusters, create some new
random clsuters at random and run the algorithm again.

To some extent this algorithm also avoids local minimum problems.  By
restarting a few times with random clusters this tend to give much
better results and is highly recommended for any implementation.

We have learned about K-means and expectation maximization.
Both need to know the number or clusters.

\subsection{Dimensionality reduction}

We will talk about linar dimensionality reduction.  We wish to find a
linear subspace on which to project the data. 

The algorithm is

\screenshot{gaussiandimensionalityfit}{Fitting a multidimensional
  Gaussian to a dataset}

\begin{itemize}
\item Calculate the \idx{Gaussian}
\item Calculate the \idx{eigenvectors} and \idx{eigenvalues} of the
  covariance matrix of the Gaussian.
\item Pick eigenvectors with maximum eigenvalues
\item Project data onto your chosen eigenvectors
\end{itemize}

This is standard statistics material and you can find it in many
linear algebra class.

\screenshot{eigenvectorsvalues}{Picking out the eigenvectors of a dataset}

Dimensional reduction may look a bit weird in two dimensions, but for
high dimensions it's not weird at all, just look at the eigenfaces  (fig.):

\screenshot{mitslide}{A slide created by Santioago Serrano from MIT
  containing a set of faces, used to detect ``eigenfaces'' that
  describe a low(ish) dimensional structure spanning this set of faces}

There is a smallish subspace of pictures that represents faces.  This
is the \idx{principal component analysis} (\idx{PCA} thing.

\screenshot{eigenfaces}{Eigenenfaces based on the dataset in \ref{eigenvetorsvalues}}

In figure \ref{eigenfaces} we see first (upper left) the average face,
then the eigenfaces associated with the image database.   The faces
are then projected down on the space of eigenfaces. Usually a dozen or
so eigenfaces suffice, and if the pictures are 2500 pixels each, we
have just mapped a 1500 dimensional space down to a dozen (twelve) or so
dimensional space.

Thrun has used eigenvector methods in his own research.  He scanned
people (thin/fat, tall, etc.) in 3D and to determine if there is a
latent lower dimensional space that is sufficient to determine the
different physiques and postures people can assume.

It turns out that that this is possible.  Only three dimensions
suffice to work on physique. Our surfaces are represented by tens of
thousands of data points.  The method is called ``scape'' 

\screenshot{scanexample1}{An example of body postures scanned with a
  laser scanner.  These bodies and postures can be modelled as a surprisingly
low dimensiona vector.}

Use surface meshes. Uses sampels of pose variations from each
subject.   The model can both represent articulated motion and
non-rigid muscle deformation motion (that is really cool btw).  The
model can also represent a wide variety of body shapes.  Both shape
and pose can be varied simultaneousy   Can be used for shape
completion to estimate the entire shape.  E.g. if we only have a scan
of the back of a person, we can estimate the front half.

Mesh completion is possible even when neither the person nor the pose
exists in the original training set.

Shape completion can be used to complete time series.   Starting from
motion capture markers, the movement of a ``puppet'' can be synthesized.

The trick has been to define picewise linear og even nonlinear patches
that the data is matched to.  This is not dissimilar to K-nearest
neigbour methods, but common methods used today are \idx{local linear
  embedding} and \idx{iso map} methods.  There is tons of info on
these methods on the world wide web.

\subsection{Spectral clustering}

\screenshot{bananaclusters}{Banana clusters. Not ideal for the K-Means
cluster algoritm.  Affinity clustering works much better}

Both EM and K-means will fail very badly to cluster data sets like the
one in fig \ref{bananacluster}.  The difference is the affinity, or
the proximity that affects the clustering, not the distance between
points. \idx{Spectral clustering} is a method that hellp us.

Data is stuffed into an affinity matrix where everything is mapped to
the ``affinity'' to the other points.  Quadratic distance is one option.

\screenshot{affinitymatrix}{An affinity matrix}

The \idx{affinity matrix} is a  a \idx{rank deficient matrix}.  The
vectors correspoinding to items in  the same clusters are very similar
to each other, but not to the vectors representing  elements in the
other clusters.  This kind of situation is {\em easily} addressed by
\idx{principal component analysis} (PCA).  We just pop the affinity matrix
in and get a set of eigenvectors that represents the clusters, project
each individual point to the cluster with the eigenvector it is less
distanced from and we're done. Mind you, we should only use the
eigenvectors with large eigenvalues since the eigenvectors with low
eigenvalues represents noise (most likely).

\screenshot{pca}{The principal component analysis (PCA) algorithm does
its magic}

The dimensionality is the number of large eigenvalues.   This is
called \idx{affinity based clustering} or \idx{spectral clustering}.

\screenshot{spectralclustering}{Spectral clustering.  Clustering based
on closeness to an eigenvector.}

\section{Supervised v.s. unsupervised learning}

\screenshot{learningvenn}{Venn diagram of supervised and unsupervised
  learning algorithms}

Unsupervised learning paradigm less explored than supervised
learnning.  Getting data is easy, but getting labels is hard. It's one
of the most interesting research problesems.

In between there are \idx{semi-supervised} or \idx{self supervised}
learning methods, and they use elements of both supervised and
unsupervised methods. The robot Stanley (fig \ref{stanley} on page
\pageref{stanley})  used its own sensors to generate labels on the
fly.  

\chapter{Representation with logic}

AI push agiainst complexity in thrree direction.   Agents,
environments, and also representations.

In this section we'll see how logic can be used to model the world.

\subsection{Propositional logic}

We'll start with an exmple:  We have  a world with a set of propositons:

\begin{itemize}
\item B: burglary
\item E: earthquake
\item A: The alarm going off
\item M: Mary calling
\item J: John calling.
\end{itemize}

Tese proposition can be either true or false.  Oure beliefs in these
statements are either true or false, or unknown.  We can 
combine them using logical operators:

\[
   (W\vee B) \implies A
\]

meaning: ``Earthquake or burglary implies alarm''.  We can also write
``alarm implies either john or mary calls'' as \(A\Rightarrow(A\wedge N\),

Biconditional: \(\Leftrightarrow\) means that implications flows in
both directions.

Negation: \(J\Rightarrow\neg M\).  

A propositional value is true or false with respect to a \idx{model},
and we can determine truth bi evaluating if the statement is or isn't
true as dictated by the model.

\screenshot{truthtable}{A truth table}


One method is to construct truth tables (as in fig \ref{truthtable}).
Lesson from rmz: Don't try to calculate too large truth tables by
hand. It always ends up in tears :-)  Write a small program, it's
in general both simpler and safer :-) (the exception is when some
clever trick can be applied by hand, so it's alway smart to try to be
clever, but then to give up fairly quickly and use brute force).

A \idx{valid sentence} is true in any model.  A \idx{satisfiable
  sentence} is true in some models, but not all.

Propositional logic is av very powerful tool for what it does.
However, no ability to handle uncertainty, can't talk about events
that are true or false in the world. Neither we can't talk about
relations between objects. Finally there are no shortcuts to talk
about many thing (e.g. ``every location is free of dirt''), we need a
sentence for every possibility

\subsection{First order logic}

We'll talk about FOL and propositional logic and probability theory
wrt what they say about the world (the \idx{ontological committment}),
what types of beliefs agents can have based on these logics (the
\idx{epistomological commitment}).

In FOL we have relations between objects in the world, objects and
functions on those objects.  Our beliefs can be true, false or
unknown.

In PL we have facts, but they can be only true or false.

In probability theory we have the same facts as in propositional
logic, but the beliefs can be in ranges from zero to one.

% XXX Missing screenshots for the different types of logics
% \screenshot{logics}{logics}

Another way to look at representations is to break them  into
\idx{atomic representation} and \idx{factored representation}.  Atomic has no components. They are
either identical or not (as used in the search problems we looked
at).  A factored representation has multiple facets that can be other
types than boolean.  The most complex representation is called
\idx{structured representation}.  Can include variables, relations
between objects, complex representations and relations.  This latest
is what we see in programming languages and databases (``structured
query languages'') and that's a more powerful representation, and
that's what we get in first order logic.


Like in propositional model we start with a \idx{model}   that assigns
a value to each propositional value, e.g.:

\[
\braces{P:T, Q:T}
\]

\screenshot{folmodel}{First order logic model}

In first order models we have more complex models.   We can have a set
of objects, a \idx{set of constants}, e.g. A, B, C, D , 1,2,3.  But there is
no need to have one to one correspondence referring to the same
objects.  There can also be objects without any names.  We also have a
set of  \idx{functions} that maps from objects to objects, e.g. the
numberof function:

\[
\mbox{numberof} :    \braces{A\mapsto 1, B\mapsto 3, C\mapsto 3, D\mapsto 2 \ldots}
\]


In addition to functions we can have \idx{relations}, e.g. ``above''
etc. Relations can be n-ary (unary, binary etc.) that are (can be) represented
by tuples.   There may be 0-ary relations, like ``rainy'' \mXXX{Didn't
  understand that}


The syntax of first order logic consists of sentences that statements
about relations (e.g. ``vowel(A)''). There is always equivalence (=),
in addition there are the elements from propositional logic \(\wedge,
\vee, \neg, \Rightarrow, \Leftrightarrow, () \).

The terms can be constants (uppercase) or variables (lowercase) or
function invocations.


When the quantifier is omitted we can assume a forall quantifier that
is implicit.  Usually the forall is used with a sentence that
qualifies a statement, e.g. ``for all object that are vowels, there will be only one
of them'', since the universe of all objects is so large that we
usually don't want to say much about it.

First order logic means that the relations is about objects not
relations.  \idx{Higher order} objects talk about relationships
between reations. That means that sentences in higher order logic is
in general invalid in first order logic.

When you define something, you usually want a definition with a
bidirecitonal implication.


\chapter{Planning}

Planning is in some sense the core of AI.  A* and similar algorithms
for problem solving search in a state space are a great planning
tools, but they only work when the environment is deterministic and
fully observable. In this unit we will se how to relax those
constraints.

\screenshot{blindfolded}{A blindfolded navigator will tend to follow a
mostly random walk.}

Looking at the Romania problem again.  The A* does all the planning
ahead and then executes.  In real life this doesn't work very well.
People are incapable of going in straight lines without external
references.  The yellow hiker could see shadows (it was sunny) and
that helped to make a straight line.  We need some feedback from the
environment. We need to interleave planning and execution.

In stochastic environments we have to be able to deal with
\idx{contingencies}.  \idx{Multi agent environments } means that we
must react to the other agents. \idx{Partial observability} also
introduces problems.   An example is a roadsign that may or may not
indicate that a road is closed, but we can't observe it until we're near
the sign.

In addition to properties with the environment, we may have lack of
knowledge.  The map may be incomplete or missing.  Often we need to
deal with \idx{plans that are hierarchical}.  Even for a perfectly
planned roadtrip we will still have to do some micro-planning on the
way, steering left, right, moving the pedals up and down to accomodate
curves and things that are not on the map and that we couldn't plan
for even if we don't ever deviate from the actually planned path.

Most of this can be changed by changing our point of view from
planning with \idx{world states} to instead plan from \idx{belief
  states}.


\screenshot{vacumcleaners}{Vacum world}

We're looking at the vacum world with eight states (dirt/clean, left
right).  What if the sensors of the cleaner breaks down?  We don't
which state we're in.  We must assume we are in any of the possible
worlds.  However, we can then search in the state of belief states.

\subsubsection{Sensorless planning in a deterministic world}

\screenshot{vacumbeliefstates}{Belief states in the vacum world}

If we execute action we can get knowledge about the world without
sensors.  For instance if we move right we know we are in the right
square: Either we were in the left space and moved right, or we were
in the right space and bumped against the wall and stayed there. We
know know more about the world, we only have four possible worlds even
if we haven't observed anything.   

Another interesting thing is that unlike in the world state, going
left and going right are not inverses of each other.   If we go right,
then left, we know we are in the left square and that we came there
from being in the right square.  This means that it's possible to
formulate a plan to reach a goal without {\em ever} observing the
world.  Such plans are called \idx{conformant plans}.  For example, if
our goal is to be in a clean state, all we have to do is suck.

\subsubsection{Partial sensing in a deterministic world}

\screenshot{partialobservablecleaner}{A partially observable vacum
  cleaner's world}

The  cleaner can see if it's dirt in its current location.  In a
deterministic world state the size of the belief state will stay the
same or decrease.  Observations works the oter way, we are taking the
current belief state and partition it up.   Observation of alone can't
introduce new states, but it can make us less confused than we were
before the observation.

\subsubsection{Stochastic environment}

\screenshot{stochasticenv}{A stochastic environment: Actions may or
  may not do what you think they do.}

Sometimes the wheels slip so sometimes you stay in the same location
and sometimes you move when you tell the robot to move. We are
assuming that the suck action works perfectly.  Often the state space
expands.  The action will increase the uncertainty since we don't know
what the result of the action is.  That's what \idx{stochastic}
means.  However, observation still partitions the statespace.

Notation for writing plans, use tree structure not sequence of
action.  Conditionals.  In general stochastic plans work in the limit.


\screenshot{branchingplan}{A plan with alternatives notated as branches}

However plans may be found by search just as in problem solving.  In
the notation branches with connecting lines between the outgoing edges
are branches in the {\em plan}, not in the search space.  We then
search until we have a plan that reach the goal.

In an unbounded solution, what do we need to ensure success in the
limit?  Every leaf need to be a goal.  To have  bounded solution we
can't have any loops in the plan.

\screenshot{mathnotation}{A mathematical notation for planning}

Some people like trees, other like math notation. For deterministic
states things are simple expression.  For stochastic situations the
beliefe states are bigger, but the 


\subsubsection{Tracking the predict/update cycle}

\screenshot{predictupdate}{Predicting updates (whatever that means)}

This gives us a calculus of belief states.  However, the belief states
can get large.  There may be succinct representations than just
listing them all.  For instance, we could have a algebra saying
``vacum is on the right''.  that way we can make small descriptions.

There is a notation called \idx{classical planning} that is both a
language and a method for describing states.   The state space
consists of \(k\) boolean variables so here are \(2^k\) states in that
statespace.  For the two-location vacum world there would be three
bolean variables.   A world set consist of a concrete assignment to
all the variables.  A belief state has to be a complete assignment in
classical planning, however we can also extend classical planning to
have partial assignments (e.g. ``vacum in A true'' and the rest
unknown).  We can even have a belief state that is an arbitrary
formula in bolean algebra.

In classical planning actions are represented by \idx{action schema}.
They are called \idx{schema} since there are many possible actions
that are similar to each other.

A representation for  a plane going from somewhere to somewhere
else: 

\begin{verbatim}
  Action(Fly(p, x, y) -- plane p from x to y
         precond: plane(p) AND Airport(x) AND
                  airport(y) AND
                  At(p,x)
         effect:  Not At(p,x)) AND at(p, y))
\end{verbatim}

\screenshot{classicalplanning}{Classical planning}

Now, this looks like first order logic, but this is a completely
propositional world.  We can apply the schema to specific world
states.

In figure \ref{classicalplanning} we see a somewhat more elaborate
schema for cargo transports.

\subsection{Planning}

Now we know how to represent states, but how do we do planning?  Well,
one way is just to do it like we did in problem solving with
search. We start in one state, make a graph and search in the graph.
This is forward or progressive state space search.

However, since we have this representation we can also use backward or
regression search.  We can start with the goal state.  In problem
solving we had the option of searching backwards, but in this
situation we have a wide range of real goal states. What we can do is
to look at the possible actions that can result in the desirable
goals.

\screenshot{regressionsearch}{Regression search}

Looking at the action schema we find a matching goal.  Backwards
search makes sense when we're buying a book :-)

There is another option we have, and that is to \idx{search through
  the space of plans} insetead of searching through the space of
states.  We start with a start and end state.  That isn't a good plan,
so we edit it by adding one of the operators.  Add some more elements.

\screenshot{planstatesearch}{Plan state search}

In the eighties this was a popular way of doing things, but today the
most popular way of doing things is through forward search.  The
advantage of forward search is that we can use heuristics, and since
forward search deals with concrete planned states it seems easier to
come up with good heuristics.

Looking at the for-by-four sliding puzzle.  The action is:

\begin{verbatim}
  Action(Slide(t, a, b)
      pre: on(t, a) AND tile (t) AND blank(b)
           AND adjacent(a,b)
   effect: on(t,b) AND blank(A) AND not( on(t,a)) 
           AND not(blank(b))
\end{verbatim}

With this kind of formal representation we can automatically come up
heuristics through relaxed representations by throwing out some of the
prerequisites.  If we cross out the heuristic that the tile has to be
blank we get the ``\idx{manhattan distance}'', if we throw out the
requirement that the tiles has to be adjacent, then then we have \idx{the
number of misplaced tiles heuristic}.  Many others are possible, like
\idx{ignore negative effects} (take away the ``not blank b''
requirement of the effect'').  Programs can perform these editings and
thus come up with heuristics programmatically instead of having humans
have to come up with heuristics.

\subsection{Situation calculus}

Suppose we have a goal of moving all the cargo from airport A to
airport B.  We can represent this in first order logic.   This is
regular first order logic with some convensions.  Actions are
represented as objects.  Then we have situations that are also objects
in the logic, but they don't represent states but paths.  If we arrive
at the same state through two different actions, those would be
considered different in situational calculus.  There is an initial
situation, called \(S_0\).   We have a function on situations called
``result'', so that \(S' = \mbox{reslult}(s,a)\).

Instead of describing the applicable actions, situation calculus
instead talks about the actions that are possible in a state.  There
is a predicate where some precondition of state s says it is possible
to do something.  These rules are called \idx{possibility axioms} for
actions.  Here is one such axiom for the fly action:

\begin{verbatim}
    plane(p,s) AND airport(x,s) AND airport(y,s) at(p,x,s)
     =>
     possible(fly(p,x,y),s)
\end{verbatim}

There is a convention in situational calculus that predicates that can
change depending on the situation are calle \idx{fluents}.  The
convension is that they refer to a situation, and that the parameter
is the last in the list of parameters.

In classical planning we had action schema and described what changed
one step at a time. In situation calculus it turns out it's easier to
do it the other way around, it's easier to make one axiom for each
fluent that can change.

We use a convention called \idx{successor state axioms}:  In general a
successor has a precondition like this:

\[
 \forall a,s  \, \mbox{poss}{a,s} \Rightarrow \parens{\mbox{fluent is
     true}\Leftrightarrow a \, \mbox{made it true} \vee a \,
 \mbox{didn't undo it}}
\]

\screenshot{succstateax}{succstateax}

\chapter{Planning under uncertainty}

\screenshot{uncertainttyplanningvenn}{A venn diagram depicting the
  various types of planning we can use when faced with different types
of uncertainty}

This puts togheter both planning and uncertain.  This is really
important because the world is full of uncertainty. We will learn
techniques to plan robot actions under uncertainty.

We'll learn about \idx{MDP}s (\idx{Markov Decision Processes}) and \idx{partially
Observable Markov processes}  \idx{POMDP}.

Remember the agent tasks that could be classified according to this
schema:

\begin{center}
\begin{tabular}{l|l|l}
&{\bf Deterministic}&{\bf Stochastic} \\ \hline
Fully observable       & A*, Depth first, Breadth first & MDP \\
Partialy observable  &  & POMDP \\
\end{tabular}
\end{center}

Stochastic = the outcome is nondeterministic.


\subsection{Markov decision process (MDP)}

% XXX Image of markov MDP missing

% \screenshot{markovMDP}{markovMDP}

We can think of  a graph   describing a state machine.  It becomes
\idx{markov} if the choice of action becomes somewhat random.

A markov decision process is a set of states \(s_1, \ldots, s_N\),
activities \(a_1, \ldots, a_k\) and  astate transition matrix:

\[
 T(s,a,s') = P(s' | a,s)
\]

``P(s'|...)''  is the the posterior probability of state s' given
state a ans action s.

\screenshot{markovwithreward}{A markov decision process with a reward function}

We also attach a ``reward function'' to the state to determine which
states are good and which that are not.  The planning problem is now
to maximize the reward.

\screenshot{bonnrobot}{A robot made by Thrun for the Smithsonian institution}

% Actualy the smithsonian robot.

We see a lot of interestin robotd.

\screenshot{minerrobot}{A robot made by Thrun for exploring mines in Pennsylvania}

The environments are stochastic and the planning problems are very
challenging

\subsection{The gridworld}

\screenshot{gridworld}{A simple world for a robot to navigate in}

Two goal state (absorbing states).    When the wall is hit the outcome
is stochastic, in ten percent of the cases it will simply stay put,
but with a ten percent probability it will bounce left, and with ten
percent probability it will bounce right.

Conventional planning is inadequte here.   Instead we use a
a \idx{policy}, that assigns an action to every state.

The planning problem thefore becomes one of finding the optimal
policy.

\subsubsection{Stochastic environents conventional planning}

\screenshot{stochenvconventionalplanning}{Stochastic conventional planning}

We would create a tree depending on the directions we go.  The tree
would get a very high branching factor.

Another problem with the search paradigm is that the tree may become
very deep.  Essentially ``infinite loop''.  

Yet one problem is that many states are visited more than once.

A policy method overcomes all of this.

\subsection{Policy}

\screenshot {gridworldoptimalpath}{Finding an optimal path in the gridworld} 

The optimal diretion in the gridworld is a bit nontrivial, since it's
in general more important to avoid a disastrous outcome than to have a
quick path to the goal (due to nondeterminism).

We have a reward function over all states, in the gridworld we give
+/-  100 for the goal states, but perhaps -3 for all other states to
ensure that we don't end up making very long paths.

We can now state the objective of the MDP:

% Should be braes not parens

\[
   E ( \sum_{t=0}^{\infty} R_t ) \rightarrow \mbox{max}
\]



What we want to do is to find the policy that maximizes the reward.
Sometimes people puts a \idx{discount factor} (\(\gamma^t\)) into the
sum of the equation that decays future rewards relative to immediate
rewards.  This is kind of an incentive to get to the goal as fast as
possible and  is an alternative to the penalty per step indicated
above.

The nice mathematical thing about discount factor is that it keeps the
expectation bounded.   The expectation in the equation above will
always be less than \(\leq \frac{1}{1-\gamma} |R_{max}|\).

The definition of the \idx{future sum of discounted rewards} as
defined above defines a \idx{value function}:


\[
   V^\pi(s) =    E_\pi  (\sum_{t=0}^{\infty} \gamma^t\,  R_t | s_0 = s)
\]



This is the expected sum of discounted future rewards provided that we
start in state \(s\) and execute policy \(\pi\).

We can now compute the average reward that will be received given any
choice.  There is a well defined expectation and we are going to use
them a lot.  Planning not only depends on calculating these values a
lot, it will also turn out that we will be able to find better
policies as a result of this work too.

The value function is a potential function that leads from the goals
to all the other states so that \idx{hillclimbing} leads to finding
the shortest path \mXXX{This must mean that the value function is
  convex, but I don't yet see how that can generally be the case, at
  least for interesting topologies of action sequences, so
  there must be something that is left out in the description so far}.

The algorithm is a recursive algorithm that converges to the best path
by following the gradient.

\subsection{Value iteration}

This is a \idx{truely magical algorithm} according to Thrun.  It will
recursively calculate the value function to find the \idx{optimal
value function} an from that we can derive an optimal policy.

Start with a value function with a value zero everywere, then diffuse
the value from the positive absorbing state throughout the statespace.

A recursive function to calculate the value:


\[
     V(s) \leftarrow (\mbox{max}_a \gamma \sum_{s'}  P(s'|s,a) V(s')) + R(s)
\]


We compute a value recursively by taking the max of all the possible
successor  states.  This equation is called \idx{back-up}.  In terminal
states we just assign \(R(s)\).  The process described over converges,
and when it does, the we just replace the \(\rightarrow\) with an
``='', and when the equality holds we have what is called a
\idx{bellman equality} or a \idx{bellman equation}.

The optimal policy is now define: 

\[
 \pi(s) = \mbox{argmax}_a \sum_{s'} P(s' | s,a) V(s')
\]

Once the values has been backed up, this is the way to find the
optimal thing to do.

Markov decision processes are fully observable but have stochastic
effects.  We seek to maximise a reward function, the effective is to
optize the effective amortized reward.  The solution was to use value
iteration where we are using a value.

\screenshot{markovframework}{Notation and terms used when describing
  Markov decision processes}

\begin{tabular}{ll}
Fully observable: &  \(s_1, \ldots, s_N\)    \(a_1, \ldots , a_K\)\\
Stochastic: & \(P(s' | a,s)\) \\
Reward: & \(R(s)\) \\
Objective: &\(E \sum_t \gamma^t R^t \rightarrow \mbox{max}\)\\
Value iteration: & \(V(s)\) Sometimes pairs \(Q(s,a)\) \\
Converges: & \(\pi_{\mbox{policy}} = \mbox{argmax} \)\\
\end{tabular}

The key takeaway is that we make entire field of decisions, a so
called \idx{policy} that gives optimal decisions for every state.

We'll now get back to partial observability.    We'll not in this
class go into  all of the advanced technicques that has been used in
real cases (like the car etc.)

\begin{center}
\begin{tabular}{l|l|l}
&{\bf Deterministic}&{\bf Stochastic} \\ hline
Fully observable       & A*, Depth first, Breadth first & MDP \\
Partialy observable  &  & POMDP \\
\end{tabular}
\end{center}

\screenshot{gradientpomdp}{Gradient using a partialy observable Markov
decision process (POMDP)}

In MDP we also need to think about tasks with information gathering:


Now, what if we don't know where the +100 exit is.  Instead, we need
to look for a sign, and then act on that.

\screenshot{signedY}{In this ``signed Y''problem the agent don't know where the
  target is, but can gain knowledge about that by reading what's
  written on a sign at the bottom right. This environment is
  stochastic but it's also partially observable.}

In the \ref{signedY} case the optimal choice is to go south, just to
gather information.  The question becomes: How can we formulate the
question the agent must solve in order for the detour to the south to be the optimal
path?

One solution that doesn't work is to work out both of the
possibilities and then take the average. It simply doesn't work.

A solution that works is to use information space - belief space.  The
trick is to assume that the transition in belief space, when reading
the sign, is stochastic

\screenshot{signreadingbeliefspace}{The belief space of the  signed Y problem}

\screenshot{gradientMDPTrick}{The MDP trick of assigning a gradient to
the information space according to the estimated utility of a position
in the space, and then let the agent follow a simple gradient-descent behavior.}

The MDP trick in the new belief spce and create gradients from both of
the possible belief states, but let that flow flow through the state
where the sign is read. This means that we can in fact use value
iteration (MDP style) in this new space to find a solution to
compliated partial observable   stochastic problem.

This style planning technique is very useful in artificial
intelligence.

\chapter{Reinforcement learning}

We will learn how an agent can learn how to find an optimal policy
even if he doesn't know anything about the rewards when he starts
out.  This is in contrast to the previous section where we assumed
that there were known states that were good or bad to be in.

Backgammon was a problem Gary Tisarov at IBM worked on in the
nineties.  He tried to use supervised learning with experts
classifying states. But this was tedious and he had limited
success. Generalizing from this small number of state didn't work to
well. The next experiment let one version of the program play against
another.  The winner got a positive reward and the loser a negative
reward.  He was able to arrive at a function with no input from human
players, but was still able to perform at  a level of the very best
players in the world.   This took about two hundred thousand
games. This may sound like  a lot but it really just covers about a
trillionth of the statespace of backgammon.

\screenshot{NGshelicopter}{Andrew Ng's helicopter flying upside down :-)}

Another example is Ng's helicopter, he used reinforcement learning
to let the helicopter use just a few hours of experts and then make
programs that got really good through reinforcement learning.

The three main forms of learning:

\begin{itemize}
\item Supervised. Datapoints with a classification telling us what is  right and what is wrong.
\item Unsupervised.  Just points, and we try to find patterns
  (clusters, probability distributions).
\item Reinforcement learning: Sequance of action and state
  transitions, and at some points some rewards associated with some
  states (just scalar numbers).  What we try to learn here is the
  optimal policy: What's the right thing to do in any given state.
\end{itemize}


MDP review. Aan mdp is a set of states \(s\in S\), a set of actions
\(a\in\mbox{Actions}(s)\), a start state \(s_0\) and a transition
function to give tell os how the world evolves when do somethin:
\(P(s' |s,a)\), that is the probability that we end up in state \(s'\)
when we are in state \(s\) and apply action \(a\).  The transition
function is denoted like this: \(T(s,a,s')\).  In addition we need a
reward function: \(R(s,a,s')\), or sometimes just \(R(s'\).

To solve an mdp we find a policy , an optimal policy, that optimize
the discounted total reward:

\[
  \sum_t   \gamma^tR(s_t, \pi(s_t), s_{t+1})
\]

The factor \(\gamma\) is just to make sure that things cost more if
they are further out in time (to avoid long paths). (future rewards
count less than rewards near in time).

The utility of any state is \(U(s) = \mbox{argmax}_a \sum_{s'}
P(s'|s,a) U(s') \).  Look at all the possible actions, choose the best
one .-)


Here is where reinforcement function comes in.  What if we don't know
\(R\) and don't know \(P\)? Then we can't solve the Markov decision
problem, because we don't know what we need.

we can learn r and p by interacting with the world, or we can learn
substitues so we don't have to compute with r and p.

We have several choices:

\begin{tabular}{llll}
{\bf agent}&{\bf know}&{\bf learn}&{\bf use} \\ \hline
Utility-based agent& P& R and U&  U \\
Q-learning agent&& Q(s,a)&Q \\
Reflex-agent & & \(\pi(s)\)&\(\pi\)\\
\end{tabular}

Q is a type of utility but rather being an utility over states, it is
an utility over state/action pairs, and that tells us for any given
state/action, what is the utility of the result, without using the
utilities directly. We can then use Q directly.

The reflex agent is pure stimulus/response.  No need to model the
world.

\subsection{Passive reinforcement learning}

The agent has a fixed policy and executes that but learns about the
reward (and possibly transition) while executing that policy. E.g. on
a ship, the captain has a policy for moving around but it's your job
to create the reward funciton.

The alternative is \idx{active reinforcement learning}, and that is where we
change the behavior.  This is good because we can both explore, and to
cash in early on our learning.

\subsection{Temporal diference learning}

We move between two states and learn the difference, then once we
learn something (e.g. that a state has a +1 reward), we  propagate
that learning back into the states we have traversed to get there, so
they get somewhat good if the target state was really good, back to
the start state.  The inner loop of the algorithm is:

\begin{tabbing}
{\bf if}  \=\(s'\) is new {\bf then} \(U[s']\rightarrow r'\) \\
{\bf if} \> \(s\) is not null {\bf then} \\
      \> increment \(Ns[s]\) \\
      \> \(U[s]\rightarrow U[s] + \alpha(N_s[s]) (r  + \gamma  U[s'] - U[s])\)
\end{tabbing}

\screenshot{passiveDifferenceLearning}{Passive difference learning}

New states get default reward (e.g. zero by default).   The difference
\(U[s'] - U[s])\) is the difference in utility of states, the
\(\alpha\) is the \idx{learning rate}.    The propagation goes
relatively slowly, since we only update when we get a state with a
nonzero value, which will probably be in the next iteration.

Eventually this will propagate to the correct utilty for the policy.

This strategy takes a while to converge to real values.  

There are multiple problems with this passive approach:

\begin{itemize}
\item It takes a long time to converge.
\item We are limited byt he policy choices.
\item There may be states we haven't visited and thus don't know
  anything about (a function of the policy).
\item We may must get poor estimates due to few visits.
\end{itemize}


\subsection{Active reinforcement learning}

\idx{Greedy reinforcement learner} works the same way as the \idx{passive
reinforcement learner} but after every few utility updates, we
recompute the new optimal policy so we throw away the old \(p_1\) and
replace it with a new \(p_2\)  which is the result of solving the MDP
based on our new estimates of the utility.   We continue learning
with the new policy. If the intial policy was flawed, we will tend to
move away from it.   However, we're not guaranteed to find the
\idx{optimal policy}. Since it is greedy it will never deviate from
something it things is ok.  Particular in stochastic environments that
is bad since randomness may cause us to do stupid things.    To get
out of this rut we need to explore suboptimal policies for a while
just to make sure that we explore the utility space a bit more
thoroghly before deciding on the policy to generate.

This means that we will find a tradeoff between exploration and
execution of the currently know best policy.  One way to do that is to
sometimes do something randomly.  This actually works, but it's slow
to converge.  We need to think a bit more about this whole
\idx{exploration v.s. exploitation} thing to be effective.

In greedy polcy we are keeping track both of the utility and the
policy as well as the number of times we've visited each state. There
are several reasons why things can go wrong:

\screenshot{activeerrorsources}{Active error sources}

\begin{itemize}
\item \idx{Sampling errors:} The number of samples for a state is too low.
\item \idx{Utility errors:}We can get a bad utility since the policy is off.  We're
  underestimating the utility.
\end{itemize}



What fig \ref{activeerrorsources} suggests  is a design for an agent
that is more proactive in exploring the world when it is uncertain and
will fall back to exploiting the policy it has when it is more certain
of the world.

One way of encoding this is to let the utility of a state be some
large value \(U(s) = +R\) when \(N_s < e\), or some threshold \(e\).
When we've visited the state \(e\) times we revert to the learned
utility.  When we encunter a new state we will explore it, and when we
have a good estimate of its utility we will use that utility estimate
instead.

Exploratory learning usually does much better than the greedy
learning. It both converges faster and learns better.

\mXXX{What kind of tasks in the domain we are working can we formulate
  using policies, and hence make accesible to MDP-like learning
  algorithms?}

\[
   \pi^* = \mbox{argmax}_{a \in A(s)} \sum_{s'} P(s'|s,a) U(s')
\]

If we have all the probabilities we can apply the policy, but if we
do't then we can't calculate, and hence not apply the policy since we
don't know the probabilities.  

\subsection{Q-learning}

In Q-learning we don't need the transition model. Instead we learn a
direct mapping from states and actions to utilities.

\[
   \pi^* = \mbox{argmax}_{a \in A(s)} \sum_{s'} Q(s,a)
\]

All we have to do is to take the maximum over all possible actions.

% XXX Q-learningimage  missing 
% \screenshot{qlearning}{qlearning}

In figure \ref{qlearning} we have utilities for each of the possible
actions associated with each of the possible states, nswe, hence four
different values per state.  All Q-utilities start being zero, and
then we have an update formula that is very similar to the one we have
already seen:

\[
 Q(sa) \rightarrow Q(s, a) + \alpha\parens{R(s) + \gamma W(s', a') - Q(s,a)}
\]

\screenshot{qlearningtable}{A Q-learning table}

It has a learning rate \(\alpha\), and a discount factor \(\gamma\)

\screenshot{packmanstate}{Packman state}

For really large statespaces the approaches described above will not
be directly applicable.  The space is just to large.  The
packman-states in figure \ref{packmanstate} illustrates this.  The two
states  are bad, but they are not similar.  One thing that would be
nice would be to be able to learn from both these situations and in
some sense treat them as the same thing.

Just as we did in supervised machine learning we an use a similar
reprsentation by representing as state by a colletion of impportant
features, such as \(s = \left\{f_1, f_2, \ldots\right\} \). The features
don't have to be the exact positions on the board, they can instead be 
tings like ``distance to the nearest ghost'' or the squared
distanance, or the inverse squared distance, the number of ghost
remaining and so on.

We can then let the Q-value be defined by:

\[
    Q(,a) = \sum_i w_i \cdot f_i
\]


The task then is to learn values of these weights.  How important are
each feature.   This is good since it means that similar states have
the same values, but bad if similar states have different values.

The great thing is that we can make a small modification to the
Q-learning algorithm.  Instead of just updating \(Q(s,a) \rightarrow
\ldots\) we can update the weights \(w_i \rightarrow
\ldots\) as we update the Q-values.  This is just the same type of
thing we used when did supervised learning.  It is as we bring our own
supervision to reinforcement learning.

\section{Summary}

If we have an MDP we can calculate the optimal policy.  If we don't
know what the MDP is, we can estimate it and then solve it.  Or we can
update the Q-values.  

Reinforcement learning is one of the most interesting parts of AI.
NGs helicopters or the Backgammon player are instances of this.
There is probably a lot more to learn.


\chapter{Hidden Markov models (HMMs) and filters}

Hidden Markov models and filters is used in just about every
interesting robotic system Thrun builds.  Not so strange perhaps since
his ``job talk'' that got him into Stanford as a professor was
about this topic :-)  The techniques are applicable to finance,
medicine, robiotics, weather prediction, time series prediction,
speech, language techinques and many other.


A Hidden markov model (hmm) is used to analyze or predict time
series.  Time series with noise sensors  this is  the technique of
choice.  The basic model is a \idx{Bayes network} where  the next
state only depends on the previous state:

\[
 S_1 \rightarrow  S_2 \rightarrow \ldots \rightarrow S_N
\]

Each state also emits a sequence of measurements 

\[
Z_1 \rightarrow  Z_2 \rightarrow \ldots \rightarrow Z_N
\]

\screenshot{markovchainintro}{A markov chain where the state changes
  and measurements are made.}

and it is this type of meaurements that are used in hidden markov
models and various probabilistic filters such as \idx{Kalman filters},
particle filters and many others.

In a \idx{markov chain} the state only depends on its immediate
predecessor. What makes the model {\em hidden} is the fact that we
don't observe the markov process itself, but rather the measurement
series. 

\screenshot{markovrangefinder}{Using a markov process to estimate the
  position of a robot roving around in a museum using ultrasonic rangefinders.}

We  see the tour guide robot from the national science museum.  The
robot needs to figure out where it is.  This is hard because the robot
doesn't have a sensor that tells it where it is.  Instead it has a
bunch of rangefinders that figures out how far away objects are.  It
also has maps of the environment and can use this to infer where it
is.  The problem of figuring out where the robot is the problem of
\idx{filtering}, and the underlying model is 

\screenshot{particlefiltermapper}{A mapper based on particle filters}

The second problem is the tunnel-explorer. It basically has the same
problem except that it doesn't have a map, so it uses a
particle-filter applied to robotic mapping.  The robot transcends into
the tunnel it builds many possible paths (``particles'' or
hypothesis). When the robot gets out, it is able to select which one
of the particles is the real one and is then capable of building a
coherent map.


\screenshot{speechmarkov}{A speech recognition system using a hidden
  markov model}

A final example is speech recognition (Simon Arnfield).  We see a
bunch of oscillation.   Over time that is the speech signal.   This
signal is transformed back into letter, and it is not an easy task.
Different speakers speak differently and there may be background
noise.   Today's best recognizers uses HMMs.


\section{Markov chains}

\screenshot{weathermarkov}{A markov process for predicting the
  weather.   If it rains today, it's likely to rain tomorrow (.6), and
if it's sunny today it's likely sunny tomorrow (.8)}

If we assume there are two states, rainy or sunny.   The time
evolution follows the distribution in figure \ref{weathermarkov}.

All markov chains settle to a stationary distribution (or a limit
cycle in some cases that will not be discussed ehre).  The key to
solving this is to assume that 

\screenshot{stationarydistribution}{The stationary distribution of a
  Markov process is the probabiities as they will be after a large
  number of observations.  The trick is to assume that the probability
in the next step will be equal the probability in the current step,
use this to get a system of equations that can then be solved.}

\[
\begin{array}{lcl}
    P(A_t) &=& P(A_{t+1}) \\
   && P(A_t | A_{t-1}) P(A_{t-1}) + P(A_t | B_{t+1}) \cdot   P(B_{t-1})\\
\end{array}
\]

The latter is just the total probability applied to this case to find
the stationary distribution wil lhave at A at 2/3 and at 1/3.   The
trick is to use the first equality to solve the recursive equation.

The cool thing here is that we didn't even need the initial
distribution to get the final result. \mXXX{This simply must have
  something todo with eignenvectors, but what?}

Markov chains with this property are called \idx{ergodic}, but that's
a word that can safely be forgotten :-)  This means that the markov
chain \idx{mixes}, which means that the distribution fades over time until
disappars in the end.  The speed at which it is lost is called the
\idx{mixing speed}.

If you see a sequence of states, these can be used to estimate the
distribution of probabilities in the markov model.  We can use \idx{maximum
likelyhood} estimation to figure out these probabilities, or we can use
\idx{Laplace smoothing}.

\screenshot{markovtransitionsprobs}{Estimating transition
  probabilities using a maximum likelyhood classifier}
\screenshot{laplaciansmoothing}{Estimating transition probabilities
  using a Laplace smoothed probability}

If we use a maximum likelyhood estimation we can do a calculation like
the one in figure \ref{markovtransitionsprobs}.

One of the oddities is that there may be overfitting.  One thing we
can do is to use laplacian smoothing.

When calculating laplacian probabilities, the method is this:

\begin{itemize}
\item First set up the maximum likelyhood probability as a fraction.
\item Then add the correctives, one pseudo observation (for k = 1) to
  the numerator, and a class count corrective (multiplied by k) to the
  denominator to form the laplacian smoothed probability.
\end{itemize}

Laplacian smoothing is applicable to many types 


\subsection{Hidden markov models}
\screenshot{rainydaymodel}{The rainy / sunny / happy / grumpy - hidden
markov model}

\screenshot{happygrumpyprobs}{Calculating probabilities in the
  rainy/sunny etc HMM.}

Let's consider the rainy day model again.  We can't observe the
weather, but we can observe your mood.  The rain can make you happy or
grumpy, and sunny weather can make you happy or grumpy (with some
probability).   We can answer  questions like this based on a
combination of the bayes rule and the markov transition rules.

We can use these techniques both for prediction and for state
estimation (fancy word for estimating an internal state given
measurements).

\subsubsection{HMM and robot localization}
\screenshot{robotlocalization}{A robot that can
  sense if there is a wall in the adjacent cells and can move
  about. It has a map, and from this it can determine where it is.
  Grayness indicates estimated probability of location.}

\screenshot{robotlocalization2}{After the robot
  has moved back and forth a bit, and after having seen a
  distinguishing state}

The robot knows where north is, can sense if there is a wall in the
adjancent cell.  Initially the robot has no clue where it is, so it
has to solve a \idx{global localization problem}.  After getting more
measurements, the posterior probabilities that are consistent with the
measurements will increase, and the probabilities of states
inconsistent with the measurement will decrease.  Sensor errors can
happen  and that is factored into the model.  After having passed an
\idx{distinguishing state} the robot has pretty much narrowed down its
state estimation.

\subsection{HMM Equations.}

The HMMs are specified by equations for the hidden states and observations:

\[
\begin{array}{ccccccccc}
x_1&\rightarrow&x_2&\rightarrow&x_4&\rightarrow&\ldots \\
\downarrow&&\downarrow&&\downarrow&&\downarrow&&\\
z_1&&z_2&&z_4&&\ldots \\
\end{array}
\]

This is in fact a Bayes network

Using the concept of \idx{d-separation} we can see that if \(x_2\) and
\(z_2\) is the {\em present}, then the entire past, the future, and
the present measurement are all conditionally idependent give \(x_2\).

This structure lets us do inference efficiently.  Assume that we have
a state \(x_1\) and a measurement \(z_1\) and we wish to find the
probability of an internal state variable given a specific masurement:
\(P(x_1|z_1)\). We can set up an equation for this using Bayes rule:

\[
P(x_1|z_1) = \frac{P(x_1|z_1) P(X_1)}{P(z_1)}
\]

Since the \idx{normalizer} \(P(z_1)\) doesn't depend on the target
variable \(x_1\) it's common to use the proportionality sign

\[
P(x_1|z_1) \propto P(x_1|z_1) P(X_1)
\]

The product on the right of the proportionality sign is the \idx{basic
measurement update} of hidden markov models.   The thing to remember
is to normalize.


The other thing to remember is the \idx{prediction equation} even
though it doesn't usually have anything to do with predictions, but it
comes from the fact that we wish to predict the distribution of
\(x_2\) given that we know the distribution of \(x_1\):

\[
   P(x_2) = \sum_{x_i} P(x_1) \cdot P(x_2 | x_1) 
\]

This is just the total probability of all the ways \(x_2\) can have
been produced, and that gives us the posterior probability of \(x_2\).

Given the initial state distribution the measurement updates and the
prediction equation, we know all there is to know about HMMs. :-)

\subsection{localization example}

\screenshot{hmmlocalization}{Using a hidden markov model for
  localization: A partile filter.}
\screenshot{hmmlocalization2}{Even more particle filter}

Initially the robot doesn't know where it is.   The location is
represented as a histogram with uniform low values.  The robot now
senses the next door.  Suddenly the probabilities of being near a door
increases.  The red graph is the probability of sensing a door, the
probability is higher near a door. The green line below is the
application: We multiply the prior with the measurement probability to
obtain the posterior, and voila we have an updated probability of
where the robot is.  It's that simple.

The robot takes an action to the right.    the probabilities are
shifted a bit to the right (the convolution part) and smooth them out
a little bit account for the control noise for the robot's actuators.
The robot now senses a door again, and we use the current slightly
bumpy probability measurement as the prior, apply the measurement
update and get a much sharper probability of being in front of a
door, and  in fact of being in front of the right door.

This is really easy to implement this.  Measurement is
multiplications, and motions becomes essentially convolutions (shift
with added noise).

\subsection{Particle filter algorithm}

\screenshot{particlefilter}{Particle filter, initial state: A lot of
  particles randomly distributed both in space and direction.}
\screenshot{particlefilter2}{The particle filter from figure
  \ref{particlefilter} after a few iterations. It's no longer very
  evenly distributed field of particles definite lumps representing
  beliefs with higher probabilities.}

Robot equipped with range sensor, task is to figure out where it is.
The robot knows along the black line, but doesn't know where it is.
The trick is how to represent the belief state.  This isn't like the
discrete example where we had sun or rain or a histogram approach
where we cut the space into small bins.   In particle systems the
space is represented by collection of points or particles.  Each of
those small dots is a hypothesis of where the robot may be.  It is in
fact a location and a speed vector attached to that location. The set
of all these vectors is the belief space.  There are many many
guesses, and the density of these guesses represents the posterior
probability.   Within a very short time the range sensors even if
they are very noisy will converge into a very small set of possible
locations.

Each particle is a state.   The better the measurements fits with the
predictions of the particle, the more likely it is that the particle
survives. In fact, the survival rate is in proportion to the
measurement probability.    The measurement probability is nothing but
the concistency of the sonar range measurements with the location of
the robot.

This algorithm is beautiful, and it can be imlemented in ten lines of
code.

Initially particles are uniformly distributed.    When a measurement
is given (e.g. a door seen), the measurement probability is used to
increase the particle importance weight.  The robot moves, and it now
does \idx{resampling}.  In essence this is to ensure that particles
die out in proportion to their weight, so there will now be more
particles where the weights where high, and fewer elsewhere.  If the
robot is moving, add some movement to the density field, shifting
weights in proportion to the movement.  Particles can be picked with
replacements, single particles can be picked more than once. This is
the \idx{forward prediction step}.  We now copy the particles
verbatim, but attach new weights in proportion to the measurement
probabilities.  When we now resample, we will get a new density field
for the particles.

Particle filters work in continous spaces, and what is often
underappriciated, they use computational resources in proportion to
how likely something is.  

\subsection{The particle filter algorithm}

\begin{verbatim}
s: particle set with weights
v: control
z: measurement
{w} importance weight 
particleFilter(s,v,z) 
  S'  = Emptyset
  eta = 0
  // Resample 
  // particles with un-normalized
  // weights
  for i = 1 ... n
     // Pick a particle at random, but in
     // accordance to the importance weight
     sample j ~ {w) with replacement 
     x' ~  p(x' | v, s_j)
     w' =  p( z | x' ) // Use measurement probability
     S' = S' U {<x', w'>}
     eta += w'
  end
  // Weights now need to be normalized
  for i=1 ... n
     w_i = w_i / eta
  end
\end{verbatim}


\subsection{Pro and cons}

They are really simple to implement and work really well.

\begin{enumerate}

\item They don't work very well for high dimensional spaces.
\idx{Rao-Blackwellized} particle filters goes some steps in this
direction anyhow :-) 

\item They don't work well for degenerate conditions (one particle,
  two particles)

\item Don't work well if there is no noise in the measurement models
  or control models.  You need to remix something.  

\end{enumerate}

Thrun's self-driving cars used particle filters for mapping and for a
number of other tasks.  The reason is that they are easy to implement,
are computationally efficient and and they can deal with highly
non-monotonic probability distributions with many peaks, and that's
important because many other filters can't do that.   Particle filters
are often the method of choice for problems where the posterior is
complex.

Particle filters is the most used algorithm in robotics.  But are
useful for any problem that involves time series,  uncertainty and measurement.

\chapter{Games}

Games are fun, they define a well-defined subset of the world that we
an understand.  They form a small scale problem of the problem of
dealing with adversaries.

\screenshot{toolsforproblems}{A collection of tools for solving
  problems under varying degrees of noise, stochasticisty, adversity and
  observability,  and situations
  where they are more or less applicable}

In figure \ref{toolsforproblems} we see how well various techniques
are applicable to problems dealing with stochastic environments,
partially observable environments, unknown environments, computational
limitations and adversaries.

Wittgeinstein said that there is no set of necessary and sufficient
conditions to define what games are, rather games have a set of
features and some games share some of them, others other.  At's an
overlapping set, not a simple criterion.

Single player deterministic games are solved by searching through the
statespace.

We have a state, a player (og players), and some function that gives
us the possible actors for a player in  a state a function that says
if the state is a terminal, also we get a terminal utility function
that gives us some utility for state for a player.

Let's consider chess or checkers.  They are both determinstic,
two-player zero sum games.  Zero-sum means that the sum of the
utilities for the players is zero.

\screenshot{minimax}{The minimax algorithm will find optimal game
  trees, after some possibly very long time.}

We use a similar approach to solve this .  We make a game tree, and
alternatingly the players try to maximize or minimize the total
utility for the player at the top of the choice tree (which is the
same as maximizing utility for him/herself).  The searh tree keeps
going until some terminal states.  If max is rational then max will
then choose the terminal node with maximum utility.  

\screenshot{minmaxalgorithm}{The minimax algorthm (outlined)}

The important thing is that we have taken the utility function that is
only determined for the terminal states and the definitions of the
available action and we have used this to determine the utility of
every state in the tree including the initial state.  

The time complexity for a game tree with depth m and branchout b is
\(O(b^m)\).  The storage requirement is \(b\cdot m\).

For chess with b=30 and m = 40, it will take about the lifetime of the
universe to calculate all the possibilities.   Transforming the tree
into a graph, reducing the branching factor and the depth of the
tree will all reduce the complexity.

Alpha beta pruning puts inequalities in the result nodes for the
subtrees (e.g. ``\(\leq 2\)'', and that lets us prune away whole
subtrees  without having to evaluate them.

One way of reducing the tree size is to use an \idx{evaluation
  function} that gives us an estimate of the utility of the mode.  The
function should return a higher value for stronger positions and lower
values for weaker positions.  We can find it using experience with the
game.  In chess it's traditional to say that a pawn is worth  point,
a knight three points a rook three points a bishop five points and the
queen nine, and you can add up those points to get an evaluation
function using a weighted sum for the pieces. Positive weights for
your pieces and negative weights for the oppoents pieces.  We've seen
this idea in machine learning.   We can make as many features as we
like, e.g. ``it's good to occupy the center''  etc.   We can then use
machine learning to figure out an evaluation function. We then apply
the evaluation function to each state at the cutoff point, and then
back those estimates up as they were terminal values.

In alpha-beta pruning, \(\alpha\) is the  currently best know value
for max, and \(\beta\) is the currently best know value for min.  When
we start, the best values known are plus and minus infinity.

\screenshot{alphabetapruning}{The alpha beta pruning optimization can
  reduce the size of the min/max trees by a factor of two giving a
  very significant gain in  efficiency.}

Alpha-beta gets us from \(O(b^m)\) to about \(O(b^{m/2})\)  if we do
a good job, expanding the best nodes first getting to the cutoff
points soon.  This is perfect in the sense that the result is
identical since we just don't do work we don't have to do.

Converting a tree to a graph can be efficient. In chess the use of
memorized opening positions in \idx{opening books} is an instance of
this method. Similarly for closing books.   In the mid-game we can do
is to use the \idx{killer move heuristic}: if there is one really good
move in one part of a search tree, then try that move in sister
branches for that tree.

Reducing the cutoff height. That is imperfect, and it can get us into trouble.


\screenshot{chancevalue}{When evaluating random moves in games, it's
  the expectation value of the move that counts.}

Chance is also an element of games.  Dice or other sources of
randomness makes it beneficial to figure out how to handle randomness
:-)  We can deal with stochastic functions by letting our value
function include a chance element, then we use the expected value of
the dice. The sequence of graph expansion will be, roll dice, let max
choose, roll the dice (for all the possible values of the dice), use
the expected value and then let min chose.  Repeat.

Cutoff and evaluation functions are now the main problems less.
 



\chapter{Game theory}

Game theory and AI has grown up together, taken different paths, but
ar now merging back.  Game theory focus on turn-based games where the
opponent is adversarial.  There are two problems:  Agent design (find
the optimal policy.  The second is mechanism design:  GIven the
utility, how can we design a mechanism so that utility will be
maximized in some way if the agents act rationally.

We wish to find the optimal policy when the optimal policy depends on
the opponents policy.  We'll start by looking at the prisoner's
dilemma.

\begin{quote}

Alice and Bob are both criminals taken at the same time.   Both are
are given an offer: If you testify against your cohort, you'll get a
better deal.   If both defect, they will both get a worse deal than if
they didn't defect.  If only one defects, the one that defected will
get a better deal than without defection and a better deal than the
other guy.

\end{quote}

\screenshot{prisonersdilemma}{Prisoner's dilemma}

Alice an Bob understands what is going on, they are both rational.
This isn't a zero-sum game.   

A \idx{dominant strategy} is one for which a player does better than
any other strategy no matter what the other player does.   So the
question is, is there a dominant strategy in this game?  And the
answer is, both should testify (defect).

A \idx{Pareto optimal } outome is an outcome  such that there is no
other outcome that all other players would prefer.  In this case there
is such an outcome, and that is that both of the players refuse to
cooperate.  In other words, the dominant strategy will never lead to
the optimal outcome. How come? :-)

An \idx{equilibrium} is such that no player can do better by switching
to another strategy provided that all the other players stay the same.
There is a famous proof (by John Nash) proving that every game has at
least one equilibrium point.  The question here is which, if any is
the equilibrium for this game?  And there is, its testify/testify.

If there is a dominant stratego or a pareto optimal solution to a
game, it's easy to figure out an optimal strategy.  The game \idx{two
  finger morra} does not lend itself to that kind of easy
solution. It's a betting game.  Two playes ``even'' and ``odd''.  The
players show one or two fingers, and if the total number of fingers is
even player wins that number of dollar from the odd player, if the
number of fingers is odd then the odd player wins from the even
player.

There is no single move (\idx{pure strategy}) that is best, but there
is a \idx{mixed strategy} that is, and this uses the fact that there
is a probability distribution over the moves.

\screenshot{twofingermorra}{The game ``Two fingered morra''}

\screenshot{mixedstrategymorra}{The game ``Mixed strategy Morra''}

A mixed strategy is  to announce that the moves will be chosen with a
probability distribution.  This schema will give a parameterized
outcome.

P should choose a value of p so that the two outcomes are equal!.  In
the example of \ref{mixedstrategymorra} this would indicate p = q =
7/12.

If we feed this back into the strategy of the game.  The trick then is
to show that the utility must between a couple of limits. Those limits
are both 1/12, and that makes the game solved.

Mixed strategies gives us some curious strategies.  Revealing our
optimal strategy to our strategy is ok to reveal to our opponent, but
the actual choices (choose a or b in any particular case), that is bad
since our opponents will get an advantage over us.

There are games that you can do better in if your opponent don't
believe you are rational.


When checking a strategy, always check if there are dominant
strategies.

\screenshot{twofingermorageometry}{The geometry of the strategies of
  two-fingered morra}

The geometric view in figure \ref{twofingermorageometry} indicatest
that the two player's strategy end up being the intersection of the
two possible set of strategies.

\screenshot{simplifiedpoker}{(Very) simplified Poker}

In \ref {simplifiedpoker} we use a representation called \idx{the
  squential game format} that is particularly good at keeping track of
the belief states for what the agents know and don't know.  Each agent
don't know in which node in the tre it is (the uncertainty denoted by
the dotted line).  The game tree can be solved using an approach that
is not quite the max/min approach.  One way to solve it is to convert
this game in the \idx{extensive form} into the matrix representation
we hava already seen that is the \idx{normal form} (see fig
\ref{simplepokernormalform}).

\screenshot{simplepokernormalform}{Simple poker in ``Normal form'' notation}

In general this approach lead to exponentially large tables, but for
this game it is a tractable approach.  There are two equilibria
(denoted in boldface).  For real poker the corresponding table would
have about \(10^{18}\) states, and it would be intractable, so we need
some strategy to get down to a reasonable number of state.  

One strategy abstraction: Treating multiple states as they were really
one.  One such abstraction is to treat all suits as the same.  This
means that if no player is trying to get a flush, we can treat all
four aces as if they were identical.

Another thing we can do is to lump similar cards together, i.e. if we
hold a pair of tens, we can consider the other players cards as being
``greather than ten'', ``lower than ten'' etc.   Also we can think of
bet sizes into small medium and large.   Also instead of considering
all the possible deals, we can consider subsets of them doing
``\idx{monte carlo}'' simulations instead of calculating the entire
game tree.

This apporach an handle quite a lot with partial observability,
stochastic, sequential, dynamic and multiple adversarial agents.
However, this method is not very good for unknown actions, and also
game theory doesn't help us very much with continous games since we
have this matrix form.  Doesn't deal very well against an irrational
opponent, and it doesn't deal with unknown utility.




\subsection{Dominating strategies}

\screenshot{fedvspoliticians}{Feds v.s. politicians, a game of chance
  and domination :-)}

In figure \ref{fedvspoliticians} it at least for me wasn't at all
obvious how to solve the thing, so this is a paraphrasing of the
helpful explanation given by Norvig.   

The game itself is between politicians and the feds.  They can both
contract or expand the economy or they can do nothing.   All of these
actions have different outcomes for the different players.


\begin{itemize}
\item This game has no dominating strategy.
\item Instead of  analyzing all strategies against each other (which
  is possible), we can instead look for dominated strategies.
\item For the politicians strategy ``pol:0'' dominates strategy ``pol:-'' in
  the sense that all outcomes are strictly better for the politicians
  in the former than in the latter strategy.
\item This simplifies since the feds can simply chose fed:- since that
  is better in all respects than any of the other strategies.
\item This lead the politicians to optimze by chosing ``pol:+'', which
  then is the equilbrium.
\item The utility for both parties are three, and this is in some
  sense the pareto pessimal value, so the equilibrium si not pareto optimal.
\end{itemize}



\subsection{Mechanism design}

Also called ``game design''.  We want to design the rules of the game
so that we get a high expected utility that runs the game, for the
people who plays the game and the public at large.     

\screenshot{googlesearchengineasagame}{Considering the google search
  engine as a game}

The example that is highlighted is \ref{googlesearchengineasagame} the
Google search engine considered as a game.   Ads show up at the top,
bottom or side of the page.   The idea of mechanism design is to make
it attractive for bidders and people who want to  to respond to the
ads.  One feature you would want the auciton to have is to make it
less work for the bidders if they have a dominant strategy. It's hard
to do if you don't have a dominant strategy, so you want to put a
dominant strategy into the auction game.  
An auction is \idx{strategy proof} if you don't need to think about
what all the other people ar doing, you only have to worry about your
own strategy.  We also call this \idx{truth of revealing} or
\idx{incentive compatible}.

\subsubsection{The second price auction}

Bids come in whatever they want, whomever bids the highest wins, but
the price they pay is the offer of the second highest bidder.  A
policy that is better than another strategy everywhere, then it is
said to be \idx{strictly dominating}, if it ties in some places, then
it is called \idx{weakly dominating}.

\mXXX{Need to check this again. I didn't understand the second price
  auciton thing!!}

\chapter{Advanced planning}


Last time we talked about planning we left out four important things:
Time, reseources, active perception and hierarchical plans.  We have a
task network with dependencies.  

\section{Time}

The tasks has durations.  The task is
to figure out a schedule when each of the tasks start so that we can
finish as soon as possible.  Each task has a time called ``ES'',
whichis the earliest possible start time, and another time called
``LS'' which is the latest possible start time, both of these
constrained by minimizing the total time we have to complete the
network.

We can find this using recursive formulas that can be solved by
dynamic programming.

\[
\begin{array}{lcl}
\mbox{ES}(s) &=& 0\\
\mbox{ES}(B) &=& \mbox{max}_{A\rightarrow B} \mbox{ES}(A) +\mbox{duration}(A)\\
\mbox{LS}(f) &=& \mbox{ES}(f)  \\
\mbox{LS}(B) &=& \mbox{min}_{A\leftarrow B} \mbox{LS}(B)  -
\mbox{duration}(A)\\
\end{array}
\]

\section{Resources}

\screenshot{assemblytask}{An assembly task.}
\screenshot{assemblytaskwithresources}{An assembly task with resources}


The assemblytask can't be solved because we are missing a nut, but we
have to compute \(4! \cdot 5! \) paths to discover that.  The reason
is that we would need to check all combinations of nuts and bolts
during backtracking.   The idea of resources is to let each of the
nuts and bolts be less unique and then to extend the classical
planning problem to handle this problem.  We do this by ading a new
type of statement stating that there are resources, and how many there
are.  The actions have ``consume'' clauses, and ``use'' clauses.   The
use clause uses some resource while the processing is being done, but
then returns it to the pool afterwards.   The consume clause removes a
resource from the pool never to return it.


Keeping track of resources this way gets rid of that computational or
exponential explosion by treating all of the resources identically.

\subsection{Hiearchical planning}


The idea is to reduce the abstraction gap.    What does this mean?
Well, we live about a billion seconds, each second perhaps a thousand
muscles we can operate, maybe ten times per second, so we get
something like \(10^13\) actions we can take during a lifetime (give
or take a few orders of magnitudes).  There is a big gap between that
and about \(10^4\) which is the max number of items current planning
algorithms can handle.  Part of the problem there is such a gap is
that it' really hard to work with all of the detailed muscle
movements, we'd rather work with more abstract actions. So we'll
introduce the notion of a \idx{hierarchical task network} (\idx{HTN}),
so instead of talking about all the low level steps we can talk about
higher order steps of which there is maybe a smaller number.  This
idea is called \idx{refinement planning}. Here is how it works.  In
addition to regular actions we have \idx{abstract actions} and various
ways to map these into \idx{concrete actions}.

\screenshot{refinementaction}{Hierarchical planning, refining actions}

When do we know when we have a solution?  We have a solution when at
least one of the refinements of an abstract action reaches the goal.

In addition to do an and/or search we can solve an abstract planning
problem without going down to the concrete steps.   One way to do that
is to use the concept of \idx{reachable states}.

\screenshot{reachablestate}{Reachable states}

In figure \ref{reachablestate} the dotted lines surround concrete
actions and is in itself interpreted as an abstract action.  This is
like a belief state where we are in multiple states because we don't
know which action was actually taken, but instead of being subject to
a stochastic environment we are using abstractions to quantize over
sets of states where we haven't chosen the refinement left.   We can
check if there is an intersection between the reachable state and the
goal state, and if there is the plan is feasible.  To find a plan that
actually works we should the search backwards instead of forwars in
order to have a smaller tree to search through.


\screenshot{upperlowerboundstates}{Upper and lower bounded states}

Sometimes it's very hard to specify exactly which states are
reachable, instead it is practical to work with sets of states that
are \idx{approximately reachable states}.  We can then approximate the
states with a lower bound and upper bounds of the states we might
reach, but we're not entirely certain about all the combinations.

\screenshot{sensingplanning}{Sensing and planning}



\chapter{Computer vision}

Computer vision is the first application of AI that we will look at.
Computer vision is the task as of making sense of computer imaging.
It's basic stuff. We will do some things about classification and 3D
reconstruction

\screenshot{documentcamera}{The document camera used by Thrun imaged
  by his phone}
\screenshot{pinholecamera}{A pinhole camera}

The science of how images is captured is called \idx{image
  formation}.  The simplest camera is called a \idx{pinhole camera}.
There is some basic math. governing a  pinhole camera.

\idx{Perspective projection} means that the  projective size of any
object scales with distance.   The size of the projected  image is
proportional to the size of the object being imaged, and inversely
proportional to the object being imaged:

\[
      x = X \frac{f}{Z}
\]


\screenshot{vanishingpoint}{Vanishing points are where parallell lines
in the scene being imaged converge to a single point in the image.}

Actual camera images have two dimensions, and the projection laws
applies to both dimensions.  A consequence of this is that parallell
lines seems to converge in the distance,  the point at which they meet
called a \idx{vanishing point}.  In some cases (like in fig
\ref{vanishingpoint2}) there are more than one vanishing point in a scene.

\screenshot{lens}{A lens and some equations governing its behavior}

Lenses are more effective of collecting light than pinhole cameras.
There is also a limit on how small the hole can be made, since at some
small size (and smaller) something called \idx{light diffraction} will
start to blur the image.  However, using a lens will require the lens
to be focused.   How to focus a lens is governed by the equation:

\[
   \frac{1}{f} =\frac{1}{Z} =\frac{1}{z}
\]

This latter law isn't that important, but still it's there :-O

\section{Computer vision}

Things we can do with computer vision is to:

\begin{itemize}
  \item Classify objects. 
  \item 3D reconstruction (with multiple images, or stereo cameras
    etc.)
  \item Motion analysis.
\end{itemize}

In object recognitions a key concept is called \idx{invariance}: There
is natural variations of the image that don't affect the nature of the
object itself.  We wish to be invariant in our software to to these
natural variations.   Some possible variances are:

\begin{itemize}
\item Scale (the thing becomes bigger or smaller)
\item Illumination (the light source changes)
\item Rotation (turn the thing around)
\item Deformation  (e.g. a rotor that rotates)
\item Occlusion (object behind other objects)
\item View point (vantage point). The position from which one sees the object.
\end{itemize}

These  and other invariances {\em really matter} when writing computer
software.  If we succeed in eliminating changes from the objects we
look at we will have solved a major computer vision problem.

In computer vision we usually use grayscale representations since they
in general are more robust with respect to lighting variations.   A
grayscale image is a matrix of numbers usually between 0  (black) and
255 (white).

One of the things we can do in computer vision is to \idx{extract
  features}.  We can do this by filtering the image matrix.  These
programs are called \idx{feature detectors}.   

\screenshot{amsterdamgrayscale}{A grayscale image of a bridge in Amsterdam.}
\screenshot{amsterdamedges}{The image in \ref{amsterdamgrayscale}
  after being processed by an edge detector.}

Filters are in general formulated like this:

% \newcommand{\leadsto}{\ensuremath{\rightarrow}}

\[
   I \otimes g \mapsto I'
\]

where \(I\) is an image, \(I'\) is the transformed image.  \(g\) is
called the \idx{kernel}.  In general  what happens is that we use a
\idx{linear filter}:

\[
   I'(x,y) = \sum_{u,v} I(x -u, y - v) \cdot g(u,v)
\]

\screenshot{convolutionexample}{An example of convolving a 3x3 matrix with
  with an 1x1 matrix}

This is of course a \idx{convolution}.  The kernel by any dimension \((m,n)\).

\screenshot{gradientimage}{Calculating the gradient of an image using
  two convolutions and a square root}

Gradient images are combinations of horizontal and vertical edges.

\screenshot{cannyedgedetection}{The ``Canny'' edge detector.  A really
fancy edge detector}
\screenshot{edgemasks}{A collection of masks (convolution kernels)
  that have traditionally been used to detect edges }

The state of the art in edge detection is a \idx{canny edge
  detector}.  In addition to finding the gradient magnitude it traces
areas and finds local maxima and tries to connect them in a way that
there's always just a single edge.  When multiple edges meet the canny
detector has a hole.  The detector is named after professor \idx{J. C
  Canny} at \idx{U.C.  Berkeley}, and he did most of the most
impressive pieces of work on edge detection.  There are also many
commonly used masks, listed in figure \ref{edgemasks}.

Linear filters can also be applied using \idx{gaussian kernels}.   If
you convolve an image using a Gaussian kernel, we get a blurred
image.  There are some reasons why we would want to blur an image:

\begin{itemize}
\item \idx{Down-sampling}:  It's better to blur by gaussian before
  down-sampling to avoid aliasing.
\item \idx{Noise reduction}: Reduce pixel-noise.
\end{itemize}

Kernels in linear filters are associative, so that if we have a
convolution that is a combination of multiple filters, we can do:

\[
   I \otimes f \otimes g =    I \otimes (f \otimes g)
\]

\screenshot{gaussiangradientkernel}{A Gaussian gradient kernel}

So if  \(f\) is a gradient kernel, and \(g\) is a Gaussian kernel, 
then \((f \otimes g)\) will be a \idx{Gaussian gradient kernel}.


Sometime you want to find corners.   Corners are
\idx{localizable}. The \idx{harris corner detector} is a simple way to
detect corners.


\screenshot{harrisdetector}{The Harris corner detector using
  eigenvectors of subimages to find the directions to look for corners
along. It's really good.}

The trick used by the harris detector is to use an eigenvalue
decomposition.  If we get two large eigenvalues we have a corner
(rotationally invariant).  It is a very efficient way to find stable
features in high contrast images in a (quite) invariant way.

Modern feature detectors extend the Harris detector into much more
advanced features that are localizable, has unique signatures.  Some
of these are \idx{HOG} \idx{Histogram of Oriented Gradients} and
\idx{SIFT} \idx{Scale Invariant Feature Transform}. Thrun recommends
using HOG or SIFT.

\chapter{3D vision}

The idea is to make a representation of the 3D information from the 2D
images.  The range/depth/distance is perhaps the most important
thing. The question is: Can we recover the full scene from  a single
or multiple images.

\screenshot{stereodepth}{The physics and math behind 3D stero vision}

Assuming a pinhole stero-rig, and we wish to recover the depth z of a
point p.  There is a simple trick: The size of the red triangle in fig
\ref{stereodepth} is proportional to the combined small triangles
``inside the camera'' (the blue parts, moved together to form a single
triangle).  The things we know, the focal length and the baseline is
called \idx{intrinsics}, and what we wish to recover is the depth z.

\screenshot{deptfield1}{A quiz about depth fields}

\subsection{Data association/correspondence. }



Given a point in one image, where do you search for the same image in
another image?   If you correspond  the wrong points, you will ger
\idx{phantom points}: Points you think are there but which are not
there really there.   Getting correspondences right is really
important.

\screenshot{phantompoints}{Stero rigs can give rise to ``phantom
  points''.  Points that appear to be identidal in the scene but
  really are just optical illusions.}

\screenshot{ssdminimzation}{When looking for corresponding point
  between two images taken from a stereo rig, one must look a long a
  horizontal line in the image planes}

For stereo rigs the correspondence rigs will be along a line.  How to
find correspondences we can use various techniques, matching small
image patches and matching features will both work.  One can then use
a sum of squared metric to find the displacement that minimizes that
error (disparity)  and use that as a match for the displacement.   

\screenshot{ssdcalculation}{This is a calculation used to find a
  disparity between images, the objective function will then be to
  minimize to total disparity, from that we get a disparity map as
  seen in figure \ref{disparitymap}.}

\screenshot{disparitymap}{The disparities between images in stero
  pairs of   pictures  can be summarized in a new image called a
  ``disparity map''.  This map can then be used as an estimate for a
  depth field.}

\screenshot{stanleydisparity}{The self-driving car ``Stanley'' that
  won the DARPA grand challenge uses disparity maps (among several
  other techiniques) to estimate the geometry of the road around the car.}

The ssd method is very useful when searching for image templates,
e.g. when searching ror alignments.  Using a disparity map  we can
find which parts of the image is associated with large or small
disparities.   It is in fact possible to use these disparity maps to
infer depth field information about the the images.   High disparities
indicate closeness and large disparities indicates larger distance.
In some patches it is difficult to determine the disparity since there
few features to use for detection.    Disparity maps can be used in
automatic driving (see fig \ref{stanlyedisparity}).  It isn't very
informative since there are not many features in the desert.  However,
where it does something it does a fine job.

\screenshot{colorcorrespondence}{Searching for correspondences can be
  a tricky task, since there can be both displacements and occlusions
  that come into play when generating the final result.  The task of
  the correspondence map calculation is to balance the costs from
  these different sources of errors and then minimize the total cost.}

Searching for correspondence means to search along a single scan line
but it can also be maningful to look at the context in which a pixel
appears in (color and/or grayscale).  We do the shifting by minimzing
a cost function consisting of two parts, color match and occlusion.

\screenshot{costfunctionalignment}{An example of how the cost of
  alignment can be calculated.}
\screenshot{costfunctionalignment2}{Another example of how the cost of
alignment can be calculated.}

The tricky part is to compute the optimal aligment, usually it is done
using dynamic programming.   This gives an n squared algorithm which
is reasonably fast considering the alternatives.  The algorithm either
goes right diagonal.  See fig \ref{dynamicscanline}

\screenshot{dynamicscanline}{Calculating correspondences interpreted
  as a dynamic programming problem}

We evaluate the value function (\idx{Bellman function}) and trace the way
the path propagates, and that gives us the best possible aligment for
the scanlines.  This method represents the state of the art in stereo
vision.


\screenshot{stereoproblems}{Stereo vision is difficult when
  considering points on the edge of a large sphere close by, since the
leftmost point of the sphere seen by the left imager is different than
the leftmost point seen by the right imager, etc.}

There are a few things that doesn't work very well for dynamic
programming.  If you have a small object in front of a large object,
the object will appear to the left in one camera and to the right in
the other.   

Big round near objects will not let the camera see the same point at
the edge, and that can lead to problems.

\screenshot{specularreflection}{Specular reflection is another source
  of confusion for 3D image interpretation}


A final problematic issue is reflective objects.   The specular
reflection shown in figure \ref{specularreflection} will show up in
quite different areas on the camera, and reconstruction will be tricky.

\screenshot{davisrig}{A rig devised by James Davis at Honda research: It uses stereo cameras,
  but also a projector to project images to the object being scanned.
The images are known, so they can be used to disambiguate the
interpretation of what the cameras see.}
\screenshot{davispattern2}{One pattern used by the Davis rig.}
\screenshot{davispattern3}{Another pattern used by the Davis rig.}

There are a few ways to improve stereo perception.  In figure
\ref{davisrig} a rig showing two cameras and a projector that can
project various patterns onto a scene is depicted.  These patterns can
be used to reduce the ambiguity in the scenes (see
\ref{davispattern2}).  Disambiguation becomes easier. In
\ref{davispattern3} we see how Sebastian is imaged in the rig.
Another solution is the microsoft kinect.  It uses a camera system
with a laser that adds texture to a scene.  It's actually pretty good ;-)

There are a bunch of techniques for sensing range.  Laser rangefinders
use travel time for laser pulses.   Laser rangefinders is used as an
alternative to stereo vision because it gives extremely good 3d maps

\section{Structure from Motion}

\screenshot{structurefrommotion}{``Structure from motion'' refers to
  the process of combining multiple images from a single scene into a
  single 3D representation of the scene.  The source of images can be
  either from the same time, but taken by multiple cameras, of from
  the same camera that has taken multiple images of the same scene.}

The structure refer to the 3d world.  Motion refers to the location of
the camera. The idea was to move the camera around and then recover
the 3d scene from many images.

\screenshot{sceneinterpretation}{Interpreting a scene}

Tomasi and  Kanade  in 92 used harris corner extraction and recovered
the structure. Used PCA.   Also used flight recorded stuff to produce
elevation models of the terrain.  Marc Pollifeys did some intersting
work on buildings in his own hometown.  Was able to make models from
houses.  Very impressive work.  Did maps of entire cities.  There are
a bunch of occlusion maps, but that's ok.  Everything can't be
reconstructed.  

\screenshot{sfmequation}{A snapshot of an equation used in real SFM
  systems.   The interpretation task is to minimize the error
  generated by fitting a model to the observed images through this equation.}

The math behind SFM are involved (see \ref{sfmequation}).  It uses the
perspective model, assumes moving cameras.  Contains three rotation
matrices, offset matrices etc.   The equation shows how the imaging
happens from cameras that are translated and rotated to any point in
space.   The problem of solving SFM can then be formulated as a
minimzation problem (see \ref{sfmminimization}) that can then be
solved using misc techniques \idx{gradient descent}, \idx{conjugate
  gradient}, \idx{Gauss-Newton}, \idx{Levenber Marquard} (a common
method) also using \idx{singular value decomposition (affine,
  orthographic)}.  This is heavy stuff, and also really cool :-)

\screenshot{noofvariablesforsfm}{A calculation of the number of
  variables necessary for structure from motion}

Even if you are perfectly able to recover all points in a
camera/structure setup, there will still be seven variables you can't
recover (fig \ref{noofvariablesforsfm}).  These are, position (3),
direction (3) and scaling (1).   Any one of these gives a  full
one-dimensional subspace that the solution is invariant with respect to.



\chapter{Robots - self driving cars}

\screenshot{darpachallenge}{A car (not the winner) participating in
  the DARPA grand challenge competition}

Self driving cars will to make cars safer and allow more people to use
cars (elderly, disabled etc.).  Thrun has worked most of his
profesional life towards the goal of making self-driving cars.  In
this chapter we will learn a bit about how to make these self-driving
cars :)

Thrun stated working on self driving cars in 2004 when cars were asked
to drive through the mohave desert through 140 miles of desert.
Really punishing terrain.  Many people, including privates.   Most
participant failed.  Some were very large.  Most robots failed.

The first grand challenge was interesting.  Nobody made it to the
finish line.   Some were large, some were small.  No team made it
further than five percent of the course.  Some even went up in flames.
 The farthest one got was eight miles.  In some sense this was a
 massive failure :-)

He started  a class CS294 that became the Stanford racing team.
difficult mountain terrain.  After seven hours it finished the DARPA
grand challenge.  They were insanely proud.

\screenshot{mountainterrain}{The winner, ``Stanley'' driving on a
  narrow mountain road during the DARPA grand challenge.}

\screenshot{junior}{``Junior'', the second generation of self-driving
  car that participated in the DARPA urban challenge}

\screenshot{juniorsensors}{One view of the world as seen through
  ``Junior'''s sensors.}

\screenshot{juniorprobabilisticlocalization}{An illustration of the
  probabilistic localization system used by Junior.  It uses both
  particle filters and histogram filters.}

Later the team participated with \idx{junior} in \idx{DARPA urban
  challenge}.  Junior used particle and histogram filters relative to
a map of the environment.  It was able to detect other cars and
determine which size they were.

\screenshot{googleselfdrivingcar}{Google's self driving car (a Toyota Prius)}
\screenshot{sceneinterpretationforcar}{A road scene as it is
  interpreted by a car.  Notice the other vehicles (and a person) that
  is denoted by framed boxes. }

Google has a self driving car  and drives as fast as a Prius can go.

\section{Robotics}

Robotics is the science of bridging the gap between sensor data and
actions.  In perception we get sensor data and use that to estimate
some internal state. Usually the process is recursive (or iterative)
called a \idx{filter}.
The \idx{Kinematic state} is the orientation of the robot, but not really
the speed at which it is moving.  An idealized roomba will have three
dimensions.  The dynamic state  contains all of the kinematic state
pluss velocities.  Thrun likes to have  a forward speed and a yaw
rate.

The car Thrun has made he can loate the car with about 10 cm
resolution.

Monte carlo localization.  Each particle is a three dimensional
vector (six dimensional representation).   The particle filter has two
steps, a prediction step and a XXX step. We assume a robot with two
wheels with differential drive.  It's about as complex as a car. The
state is given by a vector \([a,y,\theta]^T\), and to predict the
state we need to have a simple eqeuatio to describe what the vehicle
does as time progresses a \(\Delta t\).  \(v\) is the velocity, and
\(\omega\) is the turning velocity (the differential of its wheels):

\screenshot{localizationfilter}{localizationfilter}

\[
\begin{array}{lcl}
x'         &=& x + v \cdot \Delta{t}\cdot\cos\theta \\
y'         &=& x + v \cdot \Delta{t}\cdot\sin\theta \\
\theta' &=& \theta + \omega \cdot \Delta{t} \\
\end{array}
\]

This is an approximation to the actual movement, but it is good enough
for most practical applications in robotics.

\screenshot{montecarloprediction}{Monte carlo prediction of location
  using a particle filter}

In monte carlo localization we don't add exactly, we add noise.  A
single particle gives a set of particles.   This exactly how the
prediction step is done in the self-driving cars.  We now have a set
of prediction.  The other important part of a particle filter is the
\idx{measurement step}.  Usually we will let particles survive in
proportion to the amount that they are consistent with measurements.

\screenshot{particlefiltercalculation}{Calculating the next location
  using a particle filter}

We translate the measurements to probabilities, and then we can do the filtering.

We then apply the resampling and typically draw the particles based on
the normalized weights and repeat.

\section{Planning}

\screenshot{idealizedstreetplanning}{Idealized planning of how to
  drive to some location}
\screenshot{realstreetplanning}{Example of a high level plan for
  reaching a goal}

The planning problem is at multiple levels of abstraction.  At street
level, this is basically an mdp and/or search problem.   We can use
the usual MDP algorithm but we need to take the direction of the car
into the equation.  Red is high cost, green is low.    This is value
iteration appplied to the road graph.

\screenshot{leftturndynprogramming}{Using dynamic planning in a
  gridworld type setting}
\screenshot{reallifedynamicprogramminng}{Using dynamic programming to
  plan when to shift lanes}


Self driving cars use  this a lot.

\section{Robot path planning}

\screenshot{hybridastar}{Using the hybrid \(A^*\) algorithm to plan
  travels through a maze}

\screenshot{parkinglot}{Navigating on a parking lot}

Now the world is continous :-)  The fundamental problem is that
\(A^*\) is discrete and we need continous plans in the real world.  So
the problem is: Can we find a modification of \(A^*\) that gives us
probably correct, continous plans?  The key is to modify the state
transition.  By using an algorithm called ``hybrid \(A^*\)'', we can
get correct solutions, but not necessarily the shortest ones.   Every
time we expand a grid cell we memorize the state of the vehicle (\(x',
y', \theta'\)).    The state is creating the map as it moves.  The
plan is smoothed using a quadratic smoother.

\chapter{Natural language processing}

NLP is interesting philosophically since humans define themselves by
our ability to speak with each other.  It's something that sets us
apart from all the other animals and all the other machines.
Furthermore we would be like to talk to our computers since talking er
natural.  FInally it's in terms of learning: Lots of human knowledge
is written down in form of text, not formal procedures.  If we want
our computers to be smart we need them to be able to read and
understand that text.

\section{Language models}

\screenshot{languagemodels}{Models for language: Probabilistic models
  v.s. syntactic trees.}

Historically there has been two models.   The first has to do with
sequences of words, and it's the surface words themselves and it's
probabilistic.  We deal with the actual words we see.  These models
are primarily learned from data.

The other model deals with syntax trees.  They tend to be logical
rather than probabilistic.  We have a set of sentences that defines
the language.  A boolean distinction, not probabilistic.  It's based
om trees and categories that don't actually occur in the surface form
(e.g. an agent can't directly observe that the word ``slept'' is a
verb).  Primarily these types of methods has been hand coded. Instead
we have had experts like linguists that has encoded these rules in
code.   These rules are not hard-cut, one can imagine having trees and
probabilistic models of them,  methods for learning trees etc.

\screenshot{bagofwordsmodel}{A bumper sticker illustrating the ``bag
  of words'' model}

The \idx{bag of words} model is a probabilistic model where the words
don't have sequence information.  It's still useful for a lot of
interesting probabilistic algorithms.

We can move on from the bag of words models into models where we do
take sequences into account.   One way of doing that is to have a
sequence of words, introduce some notation for it, and desscribe the
probability of that sentence as a conditional probability over the sequence.

\[
\begin{array}{lcl}
   P(w_1,  w_2, \ldots, w_n) &=&6 P(W_{1:n}) \\
    &=& \Pi_i P(w_i | w_{1:i-1}) 
\end{array}
\]

If we use the  \idx{markov assumption} we make an assumption of only
caring about going back \(k\) steps.  Using this assumption we get

\[
P(W_i | W_{i:i-1})\approx P(W_i | W_{i-k:i-1})
\]

Furthermore there is something called the \idx{stationarity
  assumption} which is that \(P(W_i | W_{i-1}) = P(w_j | w_{j-1})\)

This gives us all the formalism we need to talk about these
\idx{probabilistic word sequence models}.   In practice there are many
different tricks.    One of these tricks is \idx{smoothing}.  There
will be a lot of counts that are small or zero, so using \idx{laplace
  smoothing} or some other smothing technique is often useful.  Also
there are times when we wish to augment these models with other data
than words, e.g. who the sender is, what time of day the message was
made.  There are cases where we wish to include context.   We may
wish to consider other things than words themselves.  We may wish to
use if the word ``dog'' is used as a noun in a sequence.  That is not
immediately observable,  so it's a \idx{hidden variable}.   Also we
may wish ti think at the phrase ``New York City'' as a single phrase
rather than three different words. Also we may wish to go smaller than
that and consider single letters.  The type of model we chose depend
of the application, but it is based on the idea of a probabilistic
model over sequences, as described above.

\screenshot{icecreamlunar}{A plot showing the frequency of the search
  terms ``ice cream'' and ``full moon'' as seen by the Google search
  engine.  We can see high regularity.    The spikes for the ice cream
search term was due to the introduction of the ``ice cream sandwich''
software release of the Android operating system.}

By observing language use we can predict things.  Norvik displays the
pattern of use of the words ``full mooon'' and ``ice cream'' we can
see that the search for ``ice cream'' has a 365 day cycle, and that
the search for the term ``full moon'' has a cycle corresponding to the
lunar cycle.  The blip on the right is caused by the fact that Google
introduced a version of the \idx{Android operating system} called \idx{Ice
  Cream Sandwich} and that caused the blip to happen.  That lasted a
few days then it disappeared.   

Some things we can do with language: Classification, clustering, input
correlation, sentiment analysis, information retrieval, question
answering, machine translation and  speech recognition.

\screenshot{shakespeareNgram1}{Text generated based on N-grams for N=1
and a corpus of Shakespeare texts.}
\screenshot{shakespeareNgram2}{Text generated based on N-grams for N=2
and a corpus of Shakespeare texts.}
% \screenshot{shakespeareNgram3}{shakespeareNgram3}
% \screenshot{shakespeareNgram4}{shakespeareNgram4}{shakespeareNgram4}

When considering ngram models, at n= 1 it's just a jumble of words.
At n=2 the sentences make make locally sense, at n=3 the sentences
even make a bit sense globally (see figs \ref {shakespeareNgram1},\ref
{shakespeareNgram2},\ref {shakespeareNgram3}, \ref
{shakespeareNgram4}).   At n=4 w see even longer structures that makes
sense.  Some sentences that are not in Shakespeare but some that are
there too.

\screenshot{ngramtest}{A quiz where I was supposed to guess which
  N-gram model had generated the various  texts.  I only got two
  wrong, no. 3 and no. 7.  Cool :-)}

Btw, in the ngramtest (fig \ref{ngramtest}) I only exchanged nr. 3 and
nr. 7.  A bit weird that I should do so well.

\screenshot{languagedetermination}{Determining which language one sees
based on individual character 2-grams}

Single letter n-grams is actually quite useful to figure out which
language a text originates in.


\section{Classification into language classes}

\screenshot{wordclasses}{Classifying texts in to word classes.}

\screenshot{gzipAsClassifier}{Using the gzip compression program as a
  classifier algorithm}

Classification into semantic classes. How to do that?   We can look
memorize common parts (``steve,'' ``andy'' for names, ``new'' and
``san'' for places, what's the last character, the last two characters
etc.)  we can then throw those into a classification algorithm :-)  

One surprise is that the ``gzip'' command is a pretty good
classifier.  Compression and classification is actually pretty related
fields.



\section{Segmentation}

\screenshot{segmentation}{Segmenting characters into words is not very
important in natural language processsing, but it certainly is
importan in optica character recognition and in speech recognition.}

In Chinese words don't have   punctuations that separete them out.  In
english text we do.  However in URL  sequences and speech we certainly
have that.  Segmentation is an important problem.  The best
segmentation  \(S^*\) is the one that maximizes the joint probability
of 

\[
 S^* = \max P(w_{1:n}) = \max \Pi_i p(w_i|w_{i:i-1})
\]

We can approximate this by making the naive Bayes assumption and treat
each word independently so we want to maximize the probability of each
word regardless of which words came be for it or comes after it.

\[
  S^* \approx \max P(w_i)
\]

That assumption is \idx{wrong} but it will turn out to be good enough
for learning.   Now, there are a lot of possible segmentations.  For a
string of length \(n\) there are \(2^{(n-1)}\) possible
segmentations.  That's a lot.  We don't want to enumerate them all.
That is one of the reason why making the naive Bayes assumption is so
useful, since it means that we can consider each word  one at a time.

If we consider the string as a concatenation of the first element
``f'' and the rest ``r'', the best possible segmentation is:

\newcommand{\argmax}[1]{\ensuremath{\mbox{argmax}_{#1}}}

\[
   S^* = \argmax{s=f+r} p(f) \cdot P(S^*(r))
\]

\screenshot{argmaxsequencing}{The idea behind the argmax approach, is
  to interpret the sequence of characters as the word that gives the
  highest probability of the characters forming a word.  Got that?}
\screenshot{argmaxinpython}{argmaxinpython}{The argmax algorithm
  implemented in Python}

The segmentation of the rest is independent of the segmentation of the
first word.  This means that the Naive Bayes assumption both makes
computation easier and makes learning easier, since it's easy to come
up with unigram probabilities from a corpus of text.  It's much harder
to come up with n-gram probabilities since we have  to make a lot more
guesses and smoothing since we just won't have the counts for them.

\screenshot{fourmillioncorpus}{Using a corpus of four million words,
  these are some segmentations that can be found}

How well does the segmentation algorithm work? In figure
\ref{fourmilloncorpus} there are some examples of the naive algorithm
at work after being trained on a four million (or was it billion)
words corpus.   Some texts are not so bad, but others are actually not
that good. Making a Markom assumption would improve the performance.


\subsection{Spelling corrections}

We are looking for the best possible correction, defined as:

\[
\begin{array}{lcl}
c^* &=& \argmax{c} P(c|w) \\
      &=& \argmax{c} p(w|c) p(c) 
\end{array}
\]

(in Bayes rule there is a correction factor, but that turns out to be
equal for all corrections so it cancels out).  The unigram statistics
\(p(c)\) we can get from the corpus.   The other part, the probability
that somebody typed the word ``w'' when they really ment ``c'' is
harder, since we can't observe it directly.  Perhaps we can look at
lists of spelling correction data.  That data is more difficult to
come by. Bootstrapping it is hard.  However there are sites that will
give you thousands (not billions or trillions) of misspellings.

\screenshot{editerrors}{One can measure the distance between words
  through ``edit errors'', how many transpositions, omissions,
  insertions etc. are necessary to transform one word into another.
  Words that are close in edit distance  may be misspellings of each others.}

\screenshot{spellingtable}{A probility table of spellings and some
  misspellings.  The frequencies are just made up by Norvig, but they
  are plausible.}

However, with few examples, we'll not get enough possible error
words. Instead we can look at lettor-to-lettor errors.  That means we
can build up probability tables for edits (inserts, transpositions,
deletions, additions) from words, and we can us these as inputs to our
algorithms instead. This is called \idx{edit distance}.  We can then
work with the most common edits and from that get conditional
probabilities for the various misspellings.  Just looking at unigram
posibilities actually gives 80 percent accurate spelling
correcter. With markov assumptions we can get up into the high
ninties.

\section{Software engineering}

\screenshot{htdig}{HTDIG is a search engine that uses an algorithmic
  approach to finding misspellings and similar words.  This is a
  snippet of its source code.}

\idx{HTDIG} is a great search engine and tries to figure out if words
sound equal. (www.htdig.org) (looking at the htfuzzy).  It's hard to
maintain code that's specific for the english language if you want to
make things work for many languages.  You may need to understand the
pronounciation rules for all the langages.  If you were working with a
probabilistic model, you would only need a corpus of words and
spelling errors (e.g. through spelling edits), it is easier software
engineering.  So you could say that machine learning over
probabilistic models is the ultimate in agile programmin

\chapter{Natural language processing}

\screenshot{sentencestructure}{A setence can be parsed using a
  grammar. For human languages, unfortunately, it is uncommon to have
  an unique parse for a sentence.  The common case is some degree of
  ambiguity, hence reduction of ambiguity is a central part of any
  natural language parsing.}

There is another way to parse sentences, and that is based on internal
grammatical structure of the sentences.  However, the sentence
structure is not unique.  There is in general no unique parse of a sentence

% \screenshot{simplegrammar}{simplegrammar}

The sentences comes from \idx{grammars}.  Grammars are sets of rules
that define sets of symbols that forms the language.  The language is
defined as all the legal sentences in the language.

There are terminal words, or composite parts of the grammar.  The
terminal words have no internal structure (as far as the grammar is
concerned).

There are some serious approches to the grammar approach, some  of
which are:

\begin{itemize}
\item   It's easy to omit good parses.
\item  It's easy to include bad parses.
\end{itemize}

Assigning probabilities to the trees and using word associations to
help figure out which parse is right are both viable approaches.
Making the grammar unambiguous is not.

\screenshot{universeofallpossiblestrings}{The universe of all possible
strings includes fare more strings than are valid sentences.  The
language as described by a simple grammar is rather small, denoted by
a neat square near the center of the figure.   Real human languages
are messy, denoted by ``ragged edges'' and are not easily modelled by
simple syntax models.}

If the actual grammars were neat things it would be easy to write a
good grammar.  However that isn't true, since human languages are very
messy.  A grammar for a real human language is difficult to make.  

\section{Probabilistic Context Free Grammar}

\screenshot{pfcg}{A probabilistic context free grammar is just a
  context free grammar where each of the possible branches in a syntax
production rule is assigned a probability, and where the sum of
probabilities for a single rule sums up to one. Using argmax type
argumentation, we can now disambiguate with equally legal but
uniqually probable sentences}

\screenshot{pfcg2}{A parse using the pcfg from figure \ref{pcfg}}
\screenshot{probabilisticparseexample}{Yet an example of probabilistic
parsing.}


Writing simple context free grammars is hard, but there are some
techniques that can help us.   One such techinque is called
\idx{probabilistic context free grammar (Pfc)}.  It's just to annotate
the syntax rules in the grammar with probabilies like indicated in
figure \ref{pfcg}.  The trick is that within every rule we add
probabilities to each of the branches in the grammar, and the
probabilities add up to one.   The probability of the sentence as a
whole is just the product of all the probabilities.


\screenshot{penntreexample}{An example of a parsed tree from the
  ``penn tree bank'' that consists of a myriad of parsed corpuses that
can be used as a bass for statistical language analysis.}

So where does the probabilities come from? Well, in the examples we've
seen so far it's just Peter Norvig that has made up the numbers, but
in the real world it's of course from real dta.  Run parsers across
huge swaths of sentences, parse them and figure out the probabilities
fo the various parse rules being the correct ones.  Use naturally
occurring text.   This is not necessarily very easy to do :-)   We do
need to parse the things, and check that the parses are actually the
right thing.   In the 1990's  this was what people did in natural
language processing.  Manual markup.  There are now corpuses that can
be used. One such example is the \idx{penn tree bank} from the
\idx{university of pennsylvania}, and it's a bank of trees :-)

As a field the AI people needed to decide on some standards for what
good parse trees are, that is done, and now we can just look things
up when creating probabilistic models.

\screenshot{ambiguityreduction}{Reduction of ambiguity in a parse}

\screenshot{pfgcprob}{Calculating probabilities in a sentence using probabilistic
  context free grammar.}
\screenshot{lpfgcprob}{A lexicalized probabilistic context free
  grammar (LPFG) will go one step further than the pfgs, and assign
  probabilities to individual words being associated with each
  other. For instance, how probable is it that a ``telescope'' is
  associated with a ``a man'' through which syntactic constructions?
This knowlededeg can further help in disambiguation}


The probabilities can then be used to resolve ambiguity.  In the
example \ref{ambiguityreduction} we are interested in the conditional
probability wether men or seeing are associated with teliscopes.  When
we do that we get \idx{lexicalized probabilistic context free
  grammar}.

In lfpgs we condition the rules on specific words.  Quake for instance
is an \idx{intransitive verb}, and that means that there are some
probabilities that should be zero. However, in the real world that
isn't always so.  The dictionaries are too logical and too willing to
give a clear cut language.  The real world is messier.   Various types
of smoothings needs to be made.  We do want to make the choices based
on probabilities, and we get those probabilities from the treebanks.

We then use the real data as a basis for disambiguation.

\section{How to parse}

\screenshot{parseexample}{parseexample}

How do we parse? Well, we can use the search algorithms, either from
the words, or from the sentences at the top.  We can use both \idx{top
  down} search or \idx{bottom up} parsing.  Try all the possibilities
and backtrack when something goes wrong.  Assigning probabilities can
be done in pretty much the same way.  A nice thing about this way of
doing it is that making one choice in one part of the tree doesn't
affect the other parts of the tree.

\section{Machine translation}

\screenshot{machinetranslation}{machinetranslation}

We can look at word-by word translation,  phrase level translations.
We can consider multiple phrases at the same time, or we could go to
the level of syntax. We can even go even higher and look at
representations of semantics (meaning).  We can in fact go up and down
the pyramid depicted in fig. \ref{machinetranslation} called the
\idx{Vauquais} pyramid named after the linguist Bernard Vauquais and
the translations can be done at any level or at multiple levels.

\screenshot{phrasebasedtranslation}{Phrase based translation.    The
  model gives probabilities for a sentence being a valid translation
  by looking at the probability for a legal segmentation, the
  probabilities of word for word translations and a probability for a
  ``distortion'', which in this case is just how far the phrase has to
be moved to the left or right before it finds its proper location in
the translated phrase.}

Modern translation systems do use multiple levels.  In figure
\ref{phrasebasedtranslation} we use three diffent levels:
\idx{Segmentation}, \idx{translation} and \idx{distortion} (partialy
obscured by Norvig's pen in the illustration).   The model translats
from german to english.   The segmentation is based on  a database of
phrases.  We look for coherent phrases that occur frequently, and we
get probabilities that things represents phrases, and we then get a
high probability segmentation.   Next is the translation step telling
us which probability ``Morgen'' corresponds to ``Tomorrow''.  Then
there is distortion which tells us how much we should swap the phrases
around.  We just look at the beginning and endings of the phrases.  We
measure in the german, but get the indexes in the english.  The
distortion model is just a probability over numbers, are the things
switched from the left  or to the right, and is that shift probable or
not.  For pairs of languages where the things are not swapped very
much there will be a high probability mass under zero distortion, and
low probabilities elsewhere.  For other languages it is different.

This is a very simple model for translation.  In a real translational
model we would also have a probability of the final (in this case
english) sentence.  The problem of translation is then just a problem
of search through all the possible segmentations, translations and
distortions and probabilities of the final result being a valid
sentence. Find the ones that gives the highest translation and pick
that.  The hard part is figuring out how to do this efficiently.

\appendix

\chapter{Solving two by two game with an optimal mixed strategy}

\screenshot{maxminanswer}{Question eight in homework
  six. Aka: ``The persistent bugger''}

There was one question, question eight in homework six that caused me
an inordinate amunt of trouble.  The problem was simple enough: Find
the mixed strategy probabilities for two players based on a two-by-two
standard representation for a game.  The irritating thing was that I
just couldn't get it right.  With that in mind I set out to utterly
demolish that type of problem once and for all, and this is the result
of that demolishing process.


\section{The problem}

Given the game described by the standard matrix i figure
\ref{maxminanswer}, The game is a zero-sum game, and all the numbers
in the matrix are the utilities for the maximizing player.  Since the
game is a zero sum game the utility for the other player is minus
minus the utility fot the maximizing player.  The problem is to find
the mixed-strategy probabilities for the two players and to determine
the expected utility for the maximizing player:

\[
\begin{array}{l|c|c|}
 &\nabla :1 & \nabla :2 \\
\hline
\Delta:1 & \Delta = 4 & \Delta = -5  \\ \hline
\Delta:2 & \Delta = -3 & \Delta =   5  \\
\hline
\end{array}
\]

\section{The solution}

The procedure to solve this game is:

\begin{itemize}
\item Start by assuming that there exists an optimal  mixed strategy
  for  the maximizing player and that we  are  going to find it.
\item Let \(p\) be the probability that max plays ``1'' first.   
\item To find  the optimal \(p\) we must now assume that we don't want
  to give the minimizing player an advantage, so we must choose \(p\)
  so that whatever choice the minimizing player does, max will in the
  long run be neutral with respect to min's choices.  
\item The above condition is satisfied when this equation is
  satisified:

 \[
\begin{array}{rcl}
   p\cdot(-5) +  p\cdot 4 &=& (1-p)\cdot (-3) \cdot p\cdot 5 \\
   17 \cdot p &=& 8 \\
   p &=& \frac{8}{17} \\
 (1-p) =&=& \frac{9}{17} \\
\end{array}
\]

Since the construction of this equation is what caused me the most
problems, I'm here going to describe it in painstaking detail:

\begin{itemize}
\item First assume that \(p > 1-p\), or equivalently \(p > 0.5\).
  This means that the most probable move by max is to play ``1''.  

\item Under this assumption, the minimizing player will pick the move
  that minimizes utility for max, and that is \(-5\), the expected
  utility for the most probable move is thus \(-5 \cdot p\).  Since
  probabilities has to add up to one, we must now add the expected
  utility for the one remaining quadrant in the top row, and that
  gives us \((1-p)\cdot 4\). \mXXX{This is in fact handwaving.  I
    need to grok this to a level of absolute certainty, and I don't do
    that now}

\item We then assume that \(p < 0.5\),  and do the same exercise
  for the case where we assume that max plays 2 with a better than
  even probability.   This gives us the other side of the equation.

\item Finally we set both sides of the equation to be equal, and we
  find a situation where, seen from max's perspective perspective, it
  doesn't really matter what min chooses, since the expected payoff
  is invariant with respect to the choices made by min.

\end{itemize}

\item We then do a similar exercise for the minimizing player, and
  come up with the equation:

\[
q\cdot 4 + (1-q)\cdot (-3) = (1-q)\cdot 5 + q\cdot (-5)
\]

which resolves to \(q= 10/17\).  The only thing to keep in mind while
solving this equation is that max is still a maximizing player and min
is still minimizing, so this means that in the equation above (the
``q'') equation the oposition is trying to maximize the utility, while
in the first example (``p''), the opposition is trying to minimize
utility. Otherwise it's the same thing.

\item to calculate the utility for the maximizing player, we sum the
  expectation values of all the quadrants.

\[
\begin{array}{|c|c|} \hline
\frac{8}{17} \cdot 4    & \frac{8}{17} \cdot -5\\ \hline
\frac{9}{17} \cdot  -3 & \frac{9}{17}\cdot 5 \\
\hline
\end{array}
\]

\end{itemize}


When summing up all of these, the result is \(\frac{10}{17}\) which is
twice the utility we want, which is \(\frac{5}{17}\).  Why twice?
Well, obviously because there are two rows, but I still need to grok
this in its fullness before I can claim to understand how to solve
this type of game.

\section{The crypto challenge}

\screenshot{cryptochallenge}{crypto challenge}

\begin{verbatim}
"Esp qtcde nzyqpcpynp zy esp ezatn zq Lcetqtntlw Tyepwwtrpynp hld spwo le Olcexzfes Nzwwprp ty estd jplc." |de| | f|Cl|nf|ed|au| i|ti| |ma|ha|or|nn|ou| S|on|nd|on| |ry| |is|th|is| b|eo|as| | |f |wh| o|ic| t|, | |he|h | |ab| |la|pr|od|ge|ob| m|an| |s |is|el|ti|ng|il|d |ua|c | |he| |ea|of|ho| m| t|et|ha| | t|od|ds|e |ki| c|t |ng|br| |wo|m,|to|yo|hi|ve|u | t|ob| |pr|d |s |us| s|ul|le|ol|e | | t|ca| t|wi| M|d |th|"A|ma|l |he| p|at|ap|it|he|ti|le|er| |ry|d |un|Th|" |io|eo|n,|is| |bl|f |pu|Co|ic| o|he|at|mm| |hi| | |in| | | t| | | | |ye| |ar| |s | | |. |
\end{verbatim}


%%
%% End of content 
%%{

% http://www.geometricinformatics.com/ - geometry from lidar
% http://www.velodyne.com/lidar/-- google self driving car lidar

% Cite this page.x  (- (* 88 0.8) 3)
% http://en.wikipedia.org/wiki/Bellman_equation

\printindex
\end{document}
